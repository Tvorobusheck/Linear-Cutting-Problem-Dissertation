\chapter*{Словарь терминов}             % Заголовок
\addcontentsline{toc}{chapter}{Словарь терминов}  % Добавляем его в оглавление

% В данном словаре собраны ключевые понятия и сокращения,
% используемые в работе. Каждая запись состоит из термина и его определения.

\textbf{БДМ (бумагоделательная машина)} : оборудование для производства бумажного полотна (тамбуров) из целлюлозной массы; характеризуется шириной выпускаемой продукции, скоростью производства и временем смены плотности.

\textbf{ПРС (продольно‑резательный станок)} : станок, который разрезает бумажный тамбур на конечные рулоны согласно плану раскроя; задаётся максимальной шириной обрабатываемого тамбура, количеством ножей, скоростью нарезки и временем перестановки ножей.

\textbf{БРС (бобино‑резательный станок)} : разновидность продольно‑резательного станка, предназначенная для нарезки небольших рулонов («малых» заказов) на основе мастер‑рулонов, полученных после первой стадии раскроя.

\textbf{Тамбур} : барабанный рулон бумаги, получаемый на БДМ; служит полуфабрикатом для последующей продольной резки.

\textbf{Раскрой (раскройный план)} : организация нарезки мастер‑рулона на конечные рулоны; определяется набором форматов (ширин) и последовательностью их расположения на станке.

\textbf{Формат (рулон)} : итоговый рулон после раскроя, характеризующийся шириной и принадлежностью к конкретному заказу.

\textbf{Кромка} : технологический отступ с обеих сторон мастер‑рулона, который необходимо срезать при раскрое; учитывается как отход.

\textbf{Слитый мастер‑рулон} : мастер‑рулон, полученный из нескольких маленьких рулонов при объединении слоёв; рассматривается как отдельный заказ для БДМ.

\textbf{Плотность (граммаж)} : масса бумажного полотна на единицу площади (г/м²); влияет на скорость производства и смену параметров на БДМ.

\textbf{Марка бумаги} : класс (бренд) бумаги, характеризующий состав и технологию производства; смена марки на БДМ требует дополнительного времени.

\textbf{Склад (склады рулонов)} : хранилище для тамбуров и рулонов, ограниченное по вместимости и допустимой ширине; используется для временного хранения между производством и раскроем.

\textbf{Заказ} : требование на изготовление определённого количества рулонов заданной ширины, плотности, марки бумаги, количества слоёв и срока исполнения.

\textbf{План раскроя (Cutting Plan)} : последовательность раскроев для одного ПРС; каждый раскрой содержит набор форматов и соответствующие мастер‑рулоны.

\textbf{План производства (Making Plan)} : последовательность тамбуров, которые должны быть произведены на одной БДМ.

\textbf{Глобальный план} : комплексный план, включающий планы производства на БДМ, планы раскроя на ПРС/БРС и календарь событий, описывающий порядок операций во времени.

\textbf{Событие} : элемент очереди, характеризующий начало и окончание операции по производству тамбура или раскрою рулонов; содержит информацию о типе операции, используемом оборудовании и состоянии складов.

\textbf{Очередь событий} : упорядоченный список событий, определяющий расписание работы оборудования; обеспечивает согласование операций производства и раскроя во времени.

\textbf{Мутация (Mutation)} : генетическая операция, которая изменяет план раскроя путём случайной перестановки форматов и раскроев; направлена на поиск новых комбинаций.

\textbf{Глобальная мутация (GlobalMutate)} : оператор генетического алгоритма, модифицирующий сразу все планы раскроя в глобальном плане и пересчитывающий соответствующие планы производства и расписание.

\textbf{Кроссинговер (Crossingover)} : оператор генетического алгоритма, который из пары глобальных планов выбирает лучший по критериям план для перехода в следующую популяцию.

\textbf{Генетический поиск (Genetic Search)} : итеративный стохастический алгоритм оптимизации, основанный на принципах эволюции; использует популяцию решений, операции мутации и кроссинговера для поиска оптимального плана.

\textbf{Полный перебор (Full Enumeration)} : алгоритм, перебирающий все допустимые комбинации раскроев и производственных планов; применим только для задач малой размерности.

\textbf{Популяция} : набор глобальных планов на каждой итерации генетического алгоритма; размер популяции влияет на качество решений и время расчёта.

\textbf{Целевая функция} : вектор метрик, который минимизирует генетический алгоритм; включает суммарные отходы, максимальное количество раскроев на одном станке и общее количество перестановок ножей.

\textbf{Функция приспособленности} : числовое значение целевой функции для конкретного плана; используется для сравнения и отбора решений в генетическом алгоритме.