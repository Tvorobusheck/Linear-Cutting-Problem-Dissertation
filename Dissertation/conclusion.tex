%% Согласно ГОСТ Р 7.0.11-2011:
%% 5.3.3 В заключении диссертации излагают итоги выполненного исследования, рекомендации, перспективы дальнейшей разработки темы.
%% 9.2.3 В заключении автореферата диссертации излагают итоги данного исследования, рекомендации и перспективы дальнейшей разработки темы.
%% Поэтому имеет смысл сделать эту часть общей и загрузить из одного файла в автореферат и в диссертацию:
\chapter*{Заключение}
\addcontentsline{toc}{chapter}{Заключение}

В ходе работы была разработана математическая модель и методика решения задачи
линейного раскроя в целлюлозно‑бумажной промышленности с учётом
многочисленных технологических ограничений. Задача раскроя и связанного с ним
календарного планирования относится к NP‑трудным оптимизационным задачам,
что обуславливает необходимость использования эвристических и метаэвристических методов. Основные результаты диссертации заключаются в следующем.

\begin{enumerate}
  \item Проведён детальный анализ производственного процесса ЦБК и
    сформулирована математическая модель задачи линейного раскроя,
    учитывающая структуру заказов, ограничения на бумагоделательные машины
    (БДМ) и продольно‑резательные станки (ПРС/БРС), возможности складов,
    время переналадки ножей и смены марки бумаги, требования по срокам
    исполнения.
  \item Обоснован выбор генетического алгоритма для решения задачи при
    большом числе ограничений и предложены новые операторы: генерация
    случайных глобальных планов, функции инициализации начальной
    популяции, операторы мутации и глобальной мутации, кроссинговер и
    процедура формирования календарного расписания. Даны алгоритмические
    оценки всех этапов.
  \item Разработан программный комплекс на языке Python, реализующий
    генетический поиск и метод полного перебора. Программный комплекс
    включает модуль генерации планов производства и раскроя, модуль
    построения очереди событий, графический и консольный интерфейсы для
    ввода данных и анализа результатов.
  \item Проведены вычислительные эксперименты на реальных и
    сгенерированных наборах данных. Численные исследования показали, что
    предложенный генетический подход позволяет снижать отходы и количество
    переналадок ножей, обеспечивая более эффективные планы производства и
    раскроя по сравнению с полным перебором для крупных задач. Для задач
    малой размерности метод полного перебора остаётся предпочтительным.
  \item Полученные результаты позволяют рекомендовать использование
    предложенных алгоритмов для планирования производства и раскроя на
    предприятиях целлюлозно‑бумажной промышленности, а также в смежных
    областях, где важна интеграция расписания различных типов оборудования.
\end{enumerate}

Дальнейшие исследования могут быть направлены на разработку гибридных
эволюционных алгоритмов с адаптивной настройкой параметров, ускорение
вычислений путём параллельной обработки, расширение модели за счёт учёта
стохастических факторов (перебои в поставках, изменения спроса), а также
интеграцию задач раскроя с общим планированием поставок и логистикой.

В заключение автор выражает благодарность научному руководителю и коллегам
за поддержку, помощь и обсуждение результатов, а также всем, кто
способствовал выполнению данной работы.