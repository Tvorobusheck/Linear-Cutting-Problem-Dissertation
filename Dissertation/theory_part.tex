\chapter{Теоретическая часть}\label{ch:theor_ch}
\section{Обзор литературы}\label{sec:theor_ch/literature_review}

\subsection{Классическая задача линейного раскроя и её обобщения}\label{subsec:theor_ch/csp_classic}

Задача линейного раскроя относится к числу классических задач комбинаторной оптимизации и широко применяется при планировании производства в отраслях, где исходный материал поступает в виде заготовок фиксированной длины или ширины, а выходная продукция представлена наборами требуемых размеров. К таким отраслям относятся, в частности, металлургия, деревообработка, текстильная промышленность и целлюлозно-бумажное производство. Фундаментальные формальные постановки задачи линейного раскроя предложены П.~Гилмором и Р.~Гомори \cite{gilmore1961linear, gilmore1963linear}, где задача представлена в виде модели целочисленного линейного программирования, а также разработан подход генерации столбцов, позволяющий эффективно решать задачи большой размерности за счёт построения только необходимых схем раскроя.

Дальнейшее развитие исследований привело к формированию широкого класса задач раскроя и упаковки. Для систематизации постановок и выбора методов решения существенную роль сыграла типология, предложенная Х.~Дикхоффом \cite{dyckhoff1990typology}, а также её развитие и уточнение в более поздних работах, в частности в \cite{wascher2007typology}. Указанные классификации позволяют корректно отнести прикладные варианты раскроя, включающие дополнительные технологические ограничения, к соответствующим подклассам и тем самым обосновать применимость тех или иных методов решения.

\subsection{Точные методы решения: декомпозиция и метод ветвей и цен}\label{subsec:theor_ch/exact_methods}

Для задачи линейного раскроя и её обобщений важную роль играют точные методы, основанные на декомпозиции и последовательном уточнении решения. Подходы, использующие декомпозицию Дантцига--Вульфа и генерацию столбцов, позволяют получать сильные нижние оценки и формировать решения в пространствах большой размерности. Практические аспекты построения алгоритмов типа «ветвей и цен» (ветвление в сочетании с генерацией столбцов), а также рекомендации по их эффективной реализации подробно обсуждаются в работе Ф.~Вандербека \cite{vanderbeck2000integer}. Современный обзор математических моделей и точных алгоритмов для задач раскроя и родственных задач упаковки представлен в \cite{delorme2016bin}, где рассматриваются типовые целочисленные формулировки, схемы построения раскроев и методы улучшения вычислительной устойчивости.

\subsection{Приближённые методы: эвристики и метаэвристики}\label{subsec:theor_ch/heuristics}

В прикладных постановках, возникающих в промышленности, размерность задачи и число технологических ограничений зачастую делают применение точных методов ограниченным по вычислительным ресурсам. В связи с этим широко используются приближённые методы, в том числе эвристики (конструктивные и улучшающие), а также метаэвристики, включая локальный поиск и гибридные алгоритмические схемы. В работе Т.~Урбана \cite{urban1999solving} исследованы приближённые методы решения задачи раскроя на основе адаптивного поиска и показана эффективность подобного подхода для получения качественных решений при ограниченном времени расчёта. Существенная часть современных прикладных решений основывается на комбинации конструктивной генерации схем раскроя и процедур улучшения решения.

\subsection{Потери материала при продольной резке и отраслевые постановки}\label{subsec:theor_ch/trim_loss}

Для целлюлозно-бумажной промышленности характерны постановки, ориентированные на минимизацию потерь материала при продольной резке рулонных заготовок, включая потери на обрезь и недоиспользование ширины. Обзор соответствующих постановок и методов решения, а также обсуждение практических аспектов минимизации потерь приведены в работе А.~Хинксмана \cite{hinxman1980trimloss}. В отличие от абстрактной постановки задачи линейного раскроя, отраслевые задачи минимизации потерь включают дополнительные компоненты: ограничения по настройке ножей, зависимость затрат от смены формата, наличие ограничений по складам и допустимой перепроизводству/недопроизводству, а также требования по срокам выполнения заказов.

Важным аспектом отраслевых задач является также порядок применения схем раскроя, поскольку изменение схемы ножей и форматные переходы приводят к простоям и дополнительным затратам. Поэтому на практике целесообразно учитывать не только состав схем раскроя, но и структуру последовательности их выполнения.

\subsection{Интеграция раскроя с календарным планированием и очередностью выполнения работ}\label{subsec:theor_ch/integration_scheduling}

В производственных системах с несколькими взаимосвязанными стадиями (выпуск полуфабриката, резка, складирование и отгрузка) задача раскроя тесно связана с задачей формирования производственного расписания. Учет временных ограничений и переналадок оборудования приводит к необходимости совместного рассмотрения минимизации потерь материала и минимизации затрат времени/переналадок. Пример постановок календарного планирования операций резки на нескольких параллельных машинах представлен в \cite{giannelos2001parallel}. 

Подходы, ориентированные на одновременное решение задач минимизации потерь и формирования расписания в целлюлозно-бумажной отрасли, изучались, в частности, в \cite{harjunkoski1996trimloss, westerlund1998trimloss_scheduling}, где представлены целочисленные формулировки и обсуждены вычислительные особенности их применения. Отдельно рассматриваются постановки, в которых задача раскроя дополняется ограничениями на очередность выполнения схем раскроя; пример интегрированной постановки «раскрой + очередность» представлен в \cite{yanasse2007integrated}.

\subsection{Интеграция раскроя с планированием выпуска партий и многопериодные постановки}\label{subsec:theor_ch/integration_lotsizing}

При наличии связи между выпуском полуфабриката и последующей резкой, а также при ограничениях складирования возникает необходимость интеграции задачи раскроя с задачами планирования выпуска партий и управления запасами. Такой класс постановок рассматривается в работах, посвящённых совместному планированию объёмов производства и раскроя, например в \cite{nonas2000combined}. Современная классификация интегрированных постановок «планирование партий + раскрой», а также обзор литературы приведены в \cite{melega2018review}. Для целлюлозно-бумажной промышленности известны прикладные модели, в которых целевая функция учитывает одновременно затраты производства/хранения и потери при раскрое \cite{poltroniere2016tema}.

\subsection{Выводы по обзору и место настоящей работы}\label{subsec:theor_ch/lit_conclusion}

Таким образом, в литературе подробно изучены: (i) классическая задача линейного раскроя и методы её решения, (ii) отраслевые постановки минимизации потерь при продольной резке рулонных материалов, (iii) постановки, учитывающие очередность выполнения схем раскроя и переналадки, (iv) интеграция раскроя с задачами календарного планирования и планирования выпуска партий.

Вместе с тем для задач целлюлозно-бумажного производства сохраняется разрыв между моделями, ориентированными преимущественно на минимизацию потерь материала, и моделями, учитывающими согласованное формирование производственного расписания для взаимосвязанных стадий (выпуск полуфабриката, резка, складирование). Настоящая работа направлена на сокращение указанного разрыва за счёт разработки модели и алгоритмического подхода, обеспечивающих согласование решений по раскрою и формированию расписания с учётом технологических и временных ограничений.

\FloatBarrier


\section{Классификация и постановка задачи раскроя в производственной цепочке ЦБП}\label{sec:theor_ch/problem_context}
% Что включить:
% 1) место CSP/trim-loss в цепочке БДМ -> ПРС/БРС -> склад -> отгрузка;
% 2) типы ограничений: ширина/ножи/кромка/слои/переналадки/скорости/дедлайны;
% 3) почему «просто CSP» недостаточно: нужна связка с календарным планированием и складами.


\section{Математические модели: от задачи линейного раскроя к интегрированным моделям планирования и расписания}\label{sec:theor_ch/models}
\subsection{Классическая модель задачи линейного раскроя и декомпозиция Дантцига--Вульфа}\label{subsec:theor_ch/models_classic}
% ILP по паттернам, ограничения спроса, целевая функция (отходы/число заготовок)

\subsection{Модели с учетом технологических критериев}\label{subsec:theor_ch/models_tech}
% минимизация числа паттернов/перестановок ножей, ограничения на форматные переходы, штрафы

\subsection{Интегрированные модели: планирование выпуска партий, раскрой и формирование расписания}\label{subsec:theor_ch/models_integrated}
% связность по времени, запасы/склад, многомашинность, непрерывное время/slot-based


\section{Алгоритмические подходы к решению интегрированных задач}\label{sec:theor_ch/solution_methods}
\subsection{Точные методы и гибридные схемы (метод ветвей и цен, последовательная фиксация переменных)}\label{subsec:theor_ch/sol_exact_hybrid}
% какие идеи применимы к твоей постановке, что может быть верхней/нижней оценкой

\subsection{Приближенные методы для задач большой размерности}\label{subsec:theor_ch/sol_metaheur}
% GA/локальный поиск/матэвристики, почему это оправдано при реальных ограничениях

\subsection{Событийное представление расписания как механизм учета ресурсных конфликтов}\label{subsec:theor_ch/event_based}
% сюда логично подвести твою идею очереди событий: формирование расписания без дискретизации времени


\section{Выводы по главе}\label{sec:theor_ch/conclusions}
% 5–10 строк: что показал обзор, какой пробел закрывается, что будет в главе 2 (модель) и 3 (алгоритм/эксперименты)

\FloatBarrier

