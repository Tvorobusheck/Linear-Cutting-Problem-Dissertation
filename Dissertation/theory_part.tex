\chapter{Теоретическая часть}\label{ch:theor_ch}

\section{Обзор литературы}\label{sec:theor_ch/literature_review}

Задача линейного раскроя (Cutting Stock Problem, CSP) — одна из классических задач комбинаторной оптимизации, широко применяемая в различных отраслях промышленности, таких как металлургия, деревообработка, текстильная и бумажная промышленность. Первые формальные постановки задачи линейного раскроя были предложены в работах Гилмора и Гомори \cite{gilmore1961linear, gilmore1963linear}, где была разработана модель на основе целочисленного линейного программирования и предложен метод столбцового порождения (column generation), позволивший эффективно решать крупные задачи за счет генерации только необходимых схем раскроя.

В последующие десятилетия было предложено множество модификаций и обобщений задачи, включая многомерные варианты, задачи с учетом технологических ограничений, а также задачи с несколькими типами исходного материала. Существенный вклад в развитие методов решения внесли эвристические и метаэвристические алгоритмы, такие как жадные алгоритмы, генетические алгоритмы, алгоритмы муравьиной колонии и методы поиска с возвратом \cite{vanderbeck2000integer, dyckhoff1990typology}. В работе \cite{dyckhoff1990typology} предложена типология задач раскроя и упаковки, помогающая выбирать подходящие методы решения. Важную роль сыграли работы, посвященные практическим аспектам задачи, включая интеграцию с производственным планированием и учет ограничений на минимальную длину отходов \cite{delorme2016bin, scheithauer1999practical}. В \cite{scheithauer1999practical} представлен обзор практического опыта решения задач раскроя.

Также стоит отметить вклад Урбана \cite{urban1999solving}, который предложил эффективные эвристические алгоритмы для решения задачи раскроя, основанные на адаптивном поиске, и Воронова \cite{voronov2005application}, исследовавшего применение математических методов для оптимизации раскроя в деревообрабатывающей промышленности.

Современные исследования направлены на расширение классической задачи: учёт технологических ограничений, многомерные и многорулонные задачи, интеграцию с производственным планированием и минимизацию отходов. В результате задача линейного раскроя остаётся актуальной как с теоретической, так и с практической точки зрения.

\FloatBarrier

