\chapter{Программная реализация и экспериментальное исследование}
\label{chap:experiments}

\section{Цели и методика экспериментального исследования}
\label{sec:experiment_goals}

Экспериментальное исследование направлено на верификацию работоспособности
разработанного программного комплекса и оценку его применимости к задачам
оптимизации раскроя рулонной продукции и планирования загрузки оборудования
в условиях целлюлозно-бумажного производства.

Цели экспериментального исследования:
\begin{enumerate}
    \item подтвердить корректность обработки входных данных и формирования выходных результатов в заявленных форматах;
    \item экспериментально оценить качество формируемых производственных планов по показателям потерь материала и числа переналадок;
    \item сравнить режимы оптимизационного поиска (генетический поиск и полный перебор) по качеству решения и вычислительным затратам;
    \item определить практические границы применимости полного перебора и целесообразные режимы использования генетического алгоритма.
\end{enumerate}

В рамках исследования проверяются следующие гипотезы:
\begin{itemize}
    \item[\textbf{H1}] автоматизация подготовки производственных планов уменьшает долю ручных операций и снижает вероятность ошибок, связанных с подготовкой исходных данных и построением раскройных комбинаций;
    \item[\textbf{H2}] режим полного перебора позволяет получать оптимальные решения на задачах малой размерности и может использоваться как эталон для проверки качества приближённых методов;
    \item[\textbf{H3}] режим генетического поиска обеспечивает построение сбалансированных решений на задачах большей размерности при приемлемом времени расчёта.
\end{itemize}

Методика исследования включает следующие этапы:
\begin{enumerate}
    \item подготовка входных наборов данных в формате CSV/XLSX и проверка корректности чтения и валидации структуры;
    \item запуск программного комплекса в двух режимах оптимизации:
    \begin{itemize}
        \item полный перебор (для задач малой размерности);
        \item генетический алгоритм (для задач средней и большой размерности) с фиксированными параметрами популяции;
    \end{itemize}
    \item формирование выходных данных в формате XLSX/CSV и вычисление статистических показателей решения;
    \item анализ результатов и формулирование выводов о применимости подхода.
\end{enumerate}

\section{Описание тестового стенда}
\label{sec:test_environment}

Разработанный программный комплекс реализован на языке Python и построен по модульному принципу.
В состав программной системы входят следующие основные функциональные компоненты:
\begin{itemize}
    \item модуль чтения и проверки входных данных (CSV/XLSX), обеспечивающий контроль корректности структуры и значений;
    \item модуль оптимизационного поиска, реализующий два режима: генетический поиск и полный перебор;
    \item модуль формирования результатов и расчёта статистических показателей (распределение заказов по машинам, раскройные комбинации, число ножей, потери ширины, число переналадок и др.);
    \item модуль графического интерфейса пользователя.
\end{itemize}

Графический интерфейс реализован с использованием библиотеки Tkinter.
Интерфейс поддерживает выбор входного и выходного файлов, переключение режима оптимизации,
а также настройку параметров генетического алгоритма (в частности, размера популяции) и
отображение текущего статуса выполнения. Для обеспечения отзывчивости интерфейса
выполнение вычислительной части запускается в отдельном потоке; по завершении работы
пользователь получает уведомление об успешном завершении либо о возникновении ошибки.

Ниже приведена рекомендуемая форма фиксации параметров тестового стенда
(конкретные значения следует заполнить по факту проведения вычислительных экспериментов):

\begin{table}[htbp]
\centering
\caption{Характеристики тестового стенда}%
\label{tab:test_environment}%
\begin{SingleSpace}
\begin{tabular}{p{0.35\linewidth}p{0.55\linewidth}}
\toprule
\textbf{Параметр} & \textbf{Значение} \\
\midrule
Процессор & \texttt{<указать модель CPU>} \\
Оперативная память & \texttt{<указать объём RAM>} \\
Накопитель & \texttt{<SSD/HDD, объём>} \\
Операционная система & \texttt{<Windows/Linux, версия>} \\
Версия Python & \texttt{<например, 3.10.x>} \\
Основные библиотеки & \texttt{<например, pandas/openpyxl и др.>} \\
\bottomrule
\end{tabular}
\end{SingleSpace}
\end{table}

\section{Библиотека тестовых задач}
\label{sec:test_datasets}

В качестве входных данных программный комплекс использует таблицу заказов,
в которой каждая строка соответствует отдельному производственному заказу.
Согласно принятой структуре данных, запись о заказе может включать:
ширину продукции, количество рулонов, марку бумаги, сроки выполнения,
ограничения продольно-резательных станков и бумагоделательных машин,
а также дополнительные параметры.

В рамках экспериментального исследования целесообразно рассматривать три класса наборов данных.

\subsection{Сгенерированные наборы данных}
Сгенерированные наборы данных используются для систематической проверки
поведения алгоритмов при контролируемом изменении параметров задачи.
При генерации рекомендуется варьировать:
\begin{itemize}
    \item число заказов и диапазоны ширин;
    \item распределение объёмов (количество рулонов по заказам);
    \item наличие/отсутствие дополнительных ограничений (например, по марке/срокам);
    \item число доступных машин (БДМ и ПРС) и их технологические ограничения.
\end{itemize}

Сгенерированные задачи удобно применять для определения границ применимости полного перебора:
при росте размерности число возможных комбинаций раскроя возрастает комбинаторно,
и режим полного перебора становится вычислительно неэффективным.

\subsection{Реальные производственные задачи}
Реальные производственные задачи формируются на основе фактических заказов предприятия
(в формате CSV/XLSX) и отражают характерные производственные особенности: неоднородность номенклатуры,
различие марок продукции, наличие сроков исполнения и ограничения оборудования.
Данный класс задач используется для оценки практической применимости программного комплекса,
а также для демонстрации удобства графического интерфейса и корректности формирования выходных файлов.

\subsection{Наборы малой размерности для эталонного сравнения}
Для проверки качества решений генетического алгоритма формируется подмножество задач малой размерности,
для которых возможно получение точного решения методом полного перебора.
Полученные точные решения используются как эталон при оценке качества приближённого режима.

\section{Сравнительный анализ}
\label{sec:comparison}

\subsection{Алгоритмы для сравнения}
В программном комплексе реализованы два режима оптимизационного поиска:
\begin{enumerate}
    \item \textbf{Полный перебор} — применяется для задач малой размерности и позволяет получить оптимальное решение;
    данный режим используется как средство верификации и эталонного сравнения.
    \item \textbf{Генетический алгоритм} — применяется для задач большей размерности и предназначен для получения
    качественных (сбалансированных) решений при ограниченном времени расчёта.
\end{enumerate}

Таким образом, сравнительный анализ проводится в двух плоскостях:
(i) качество решения (потери, число переналадок и др.) и (ii) вычислительные затраты
(время выполнения, устойчивость результата при повторных запусках генетического алгоритма).

\subsection{Метрики сравнения}
По результатам работы формируются выходные файлы в формате XLSX/CSV,
содержащие распределение заказов по оборудованию, раскройные комбинации,
число ножей и потери ширины, а также вспомогательные показатели.
Для сопоставления решений используются следующие метрики:
\begin{enumerate}
    \item \textbf{Потери материала} (потери ширины) — суммарная невостребованная часть ширины при раскрое.
    \item \textbf{Число переналадок ножей} — показатель сложности реализации плана на ПРС,
    отражающий число изменений конфигурации ножей между последовательными раскроями.
    \item \textbf{Время вычисления} — время работы алгоритма до получения результата.
    \item \textbf{Стабильность} (для генетического алгоритма) — разброс качества решений при повторных запусках
    с различными начальными условиями при фиксированных параметрах популяции.
\end{enumerate}

\section{Результаты экспериментов}
\label{sec:results}

Результаты работы программного комплекса формируются в виде выходного файла (XLSX/CSV),
который включает:
\begin{itemize}
    \item распределение заказов по бумагоделательным машинам и продольно-резательным станкам;
    \item наборы раскройных комбинаций (состав раскроя и количество ножей);
    \item потери ширины по каждой операции раскроя и агрегированные показатели по плану;
    \item статистические показатели, включая число переналадок ножей и итоговую «эффективность» решения.
\end{itemize}

При проведении экспериментов рекомендуется представлять результаты в табличной и графической форме:
\begin{itemize}
    \item таблицы качества решений (потери, переналадки) для каждого набора данных и каждого режима;
    \item графики зависимости качества решения и времени расчёта от числа заказов и числа машин;
    \item для задач малой размерности — сравнение результата генетического алгоритма с оптимальным решением,
    найденным полным перебором (в абсолютных значениях и/или в относительном отклонении).
\end{itemize}

\section{Анализ сильных и слабых сторон}
\label{sec:strengths_weaknesses}

\subsection{Оптимальные условия применения}
Разработанный программный комплекс целесообразно использовать в двух типовых сценариях:
\begin{enumerate}
    \item \textbf{Задачи малой размерности} — применение режима полного перебора обеспечивает получение оптимальных решений,
    что полезно для точного планирования небольших партий и для тестирования/верификации.
    \item \textbf{Задачи средней и большой размерности} — применение генетического алгоритма позволяет получать
    качественные решения за разумное время, что соответствует практическим требованиям планирования на предприятии.
\end{enumerate}

К дополнительным практическим преимуществам относятся поддержка распространённых форматов данных,
наличие графического интерфейса для технологов и инженеров, а также модульная архитектура,
позволяющая расширять функциональность программного комплекса.

\subsection{Ограничения и узкие места}
Основные ограничения связаны со свойствами используемых режимов оптимизации:
\begin{itemize}
    \item \textbf{Полный перебор} обладает экспоненциальной трудоёмкостью и применим только для задач малой размерности.
    \item \textbf{Генетический алгоритм} является стохастическим методом, поэтому качество решения может зависеть от
    параметров популяции и начальных условий; для повышения устойчивости могут потребоваться повторные запуски и настройка параметров.
\end{itemize}

Кроме того, качество результатов существенно зависит от полноты и корректности исходных данных,
в частности от корректного задания ограничений оборудования и параметров заказов.

\section{Пример практического применения}
\label{sec:practical_example}

Практическое применение программного комплекса соответствует следующему регламенту работы:
\begin{enumerate}
    \item Пользователь подготавливает входной файл заказов в формате CSV или XLSX,
    где каждая строка соответствует заказу и содержит требуемые параметры (размеры, объёмы, марка, сроки и др.).
    \item В графическом интерфейсе выбирается входной файл и задаётся путь для сохранения результата.
    \item Выбирается режим оптимизации:
    \begin{itemize}
        \item полный перебор — для небольших задач и проверки оптимальности;
        \item генетический поиск — для задач большей размерности.
    \end{itemize}
    \item При выборе генетического поиска задаются параметры (например, размер популяции),
    после чего выполняется запуск расчёта.
    \item По завершении формируется выходной файл в формате XLSX/CSV,
    содержащий распределение заказов по машинам, раскройные комбинации,
    число ножей, потери ширины и статистические показатели, включая число переналадок.
\end{enumerate}

Пример экранных форм интерфейса, а также примеры входного файла и выходного решения
целесообразно привести в приложениях к диссертации.

\section{Выводы по главе}
\label{sec:chapter3_conclusions}

В главе представлено описание программной реализации разработанного подхода и методики экспериментального исследования.
Показано, что программный комплекс поддерживает обработку входных данных в форматах CSV/XLSX,
формирование результатов в форматах XLSX/CSV со статистическими показателями, а также два режима оптимизационного поиска:
генетический алгоритм и полный перебор. Использование графического интерфейса на основе Tkinter и запуск вычислений
в отдельном потоке обеспечивают удобство применения программного комплекса в инженерной практике.

Предложенная методика экспериментов позволяет оценить качество решений по показателям потерь материала и числа переналадок,
а также определить границы применимости точного режима полного перебора и приближённого режима генетического поиска.
