\chapter{Программная реализация и экспериментальное исследование}
\label{chap:experiments}

% Комментарий: Это практическое ядро. Используйте вашу программу как экспериментальную установку.
% Объем: 40-45 страниц. Много графиков и таблиц.

\section{Цели и методика экспериментального исследования}
\label{sec:experiment_goals}

% Какие гипотезы проверяем? Что хотим доказать?

\section{Описание тестового стенда}
\label{sec:test_environment}

% Таблица: Характеристики компьютера, ОС, версии ПО

\section{Библиотека тестовых задач}
\label{sec:test_datasets}

\subsection{Эталонные задачи (OR-Library, ESICUP и др.)}
\subsection{Сгенерированные наборы данных}
\subsection{Реальные производственные задачи}

% Комментарий: Объясните, как генерировали тесты, чтобы обеспечить репрезентативность

\section{Сравнительный анализ}
\label{sec:comparison}

% КРИТИЧЕСКИ ВАЖНЫЙ РАЗДЕЛ!
% Сравните с 2-3 известными алгоритмами

\subsection{Алгоритмы для сравнения}
% 1. Стандартный жадный алгоритм
% 2. Алгоритм из известной статьи
% 3. Коммерческое ПО (если доступно)

\subsection{Метрики сравнения}
% 1. Процент использования материала (отходы)
% 2. Время выполнения
% 3. Потребление памяти
% 4. Стабильность (разброс результатов)

\section{Результаты экспериментов}
\label{sec:results}


\section{Анализ сильных и слабых сторон}
\label{sec:strengths_weaknesses}

\subsection{Оптимальные условия применения}
% При каких размерах задач, типах ограничений ваш алгоритм работает лучше всего

\subsection{Ограничения и узкие места}
% Честно укажите, где алгоритм работает хуже

\section{Пример практического применения}
\label{sec:practical_example}

% Разберите конкретный кейс: например, раскрой листов металла на детали
% с реальными размерами и ограничениями

\section{Выводы по главе}
\label{sec:chapter3_conclusions}