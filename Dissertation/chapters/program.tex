\chapter{Программная реализация и экспериментальное исследование}
\label{chap:experiments}

\section{Цели и методика экспериментального исследования}
\label{sec:experiment_goals}

Экспериментальное исследование направлено на верификацию работоспособности
разработанного программного комплекса и оценку его применимости к задачам
оптимизации раскроя рулонной продукции и планирования загрузки оборудования
в условиях целлюлозно-бумажного производства.

Цели экспериментального исследования:
\begin{enumerate}
    \item подтвердить корректность обработки входных данных и формирования выходных результатов в заявленных форматах;
    \item экспериментально оценить качество формируемых производственных планов по показателям потерь материала и числа переналадок;
    \item сравнить режимы оптимизационного поиска (генетический поиск и полный перебор) по качеству решения и вычислительным затратам;
    \item определить практические границы применимости полного перебора и целесообразные режимы использования генетического алгоритма.
\end{enumerate}

В рамках исследования проверяются следующие гипотезы:
\begin{itemize}
    \item[\textbf{H1}] автоматизация подготовки производственных планов уменьшает долю ручных операций и снижает вероятность ошибок, связанных с подготовкой исходных данных и построением раскройных комбинаций;
    \item[\textbf{H2}] режим полного перебора позволяет получать оптимальные решения на задачах малой размерности и может использоваться как эталон для проверки качества приближённых методов;
    \item[\textbf{H3}] режим генетического поиска обеспечивает построение сбалансированных решений на задачах большей размерности при приемлемом времени расчёта.
\end{itemize}

Методика исследования включает следующие этапы:
\begin{enumerate}
    \item подготовка входных наборов данных в формате CSV/XLSX и проверка корректности чтения и валидации структуры;
    \item запуск программного комплекса в двух режимах оптимизации:
    \begin{itemize}
        \item полный перебор (для задач малой размерности);
        \item генетический алгоритм (для задач средней и большой размерности) с фиксированными параметрами популяции;
    \end{itemize}
    \item формирование выходных данных в формате XLSX/CSV и вычисление статистических показателей решения;
    \item анализ результатов и формулирование выводов о применимости подхода.
\end{enumerate}

\section{Описание тестового стенда}
\label{sec:test_environment}

Разработанный программный комплекс реализован на языке Python и построен по модульному принципу.
В состав программной системы входят следующие основные функциональные компоненты:
\begin{itemize}
    \item модуль чтения и проверки входных данных (CSV/XLSX), обеспечивающий контроль корректности структуры и значений;
    \item модуль оптимизационного поиска, реализующий два режима: генетический поиск и полный перебор;
    \item модуль формирования результатов и расчёта статистических показателей (распределение заказов по машинам, раскройные комбинации, число ножей, потери ширины, число переналадок и др.);
    \item модуль графического интерфейса пользователя.
\end{itemize}

Графический интерфейс реализован с использованием библиотеки Tkinter.
Интерфейс поддерживает выбор входного и выходного файлов, переключение режима оптимизации,
а также настройку параметров генетического алгоритма (в частности, размера популяции) и
отображение текущего статуса выполнения. Для обеспечения отзывчивости интерфейса
выполнение вычислительной части запускается в отдельном потоке; по завершении работы
пользователь получает уведомление об успешном завершении либо о возникновении ошибки.

Ниже приведена рекомендуемая форма фиксации параметров тестового стенда
(конкретные значения следует заполнить по факту проведения вычислительных экспериментов):

\begin{table}[htbp]
\centering
\caption{Характеристики тестового стенда}%
\label{tab:test_environment}%
\begin{SingleSpace}
\begin{tabular}{p{0.35\linewidth}p{0.55\linewidth}}
\toprule
\textbf{Параметр} & \textbf{Значение} \\
\midrule
Процессор & \texttt{<указать модель CPU>} \\
Оперативная память & \texttt{<указать объём RAM>} \\
Накопитель & \texttt{<SSD/HDD, объём>} \\
Операционная система & \texttt{<Windows/Linux, версия>} \\
Версия Python & \texttt{<например, 3.10.x>} \\
Основные библиотеки & \texttt{<например, pandas/openpyxl и др.>} \\
\bottomrule
\end{tabular}
\end{SingleSpace}
\end{table}

\section{Библиотека тестовых задач}
\label{sec:test_datasets}

В качестве входных данных программный комплекс использует таблицу заказов,
в которой каждая строка соответствует отдельному производственному заказу.
Согласно принятой структуре данных, запись о заказе может включать:
ширину продукции, количество рулонов, марку бумаги, сроки выполнения,
ограничения продольно-резательных станков и бумагоделательных машин,
а также дополнительные параметры.

В рамках экспериментального исследования целесообразно рассматривать три класса наборов данных.

\subsection{Сгенерированные наборы данных}
Сгенерированные наборы данных используются для систематической проверки
поведения алгоритмов при контролируемом изменении параметров задачи.
При генерации рекомендуется варьировать:
\begin{itemize}
    \item число заказов и диапазоны ширин;
    \item распределение объёмов (количество рулонов по заказам);
    \item наличие/отсутствие дополнительных ограничений (например, по марке/срокам);
    \item число доступных машин (БДМ и ПРС) и их технологические ограничения.
\end{itemize}

Сгенерированные задачи удобно применять для определения границ применимости полного перебора:
при росте размерности число возможных комбинаций раскроя возрастает комбинаторно,
и режим полного перебора становится вычислительно неэффективным.

\subsection{Реальные производственные задачи}
Реальные производственные задачи формируются на основе фактических заказов предприятия
(в формате CSV/XLSX) и отражают характерные производственные особенности: неоднородность номенклатуры,
различие марок продукции, наличие сроков исполнения и ограничения оборудования.
Данный класс задач используется для оценки практической применимости программного комплекса,
а также для демонстрации удобства графического интерфейса и корректности формирования выходных файлов.

\subsection{Наборы малой размерности для эталонного сравнения}
Для проверки качества решений генетического алгоритма формируется подмножество задач малой размерности,
для которых возможно получение точного решения методом полного перебора.
Полученные точные решения используются как эталон при оценке качества приближённого режима.

\section{Сравнительный анализ}
\label{sec:comparison}

\subsection{Алгоритмы для сравнения}
В программном комплексе реализованы два режима оптимизационного поиска:
\begin{enumerate}
    \item \textbf{Полный перебор} — применяется для задач малой размерности и позволяет получить оптимальное решение;
    данный режим используется как средство верификации и эталонного сравнения.
    \item \textbf{Генетический алгоритм} — применяется для задач большей размерности и предназначен для получения
    качественных (сбалансированных) решений при ограниченном времени расчёта.
\end{enumerate}

Таким образом, сравнительный анализ проводится в двух плоскостях:
(i) качество решения (потери, число переналадок и др.) и (ii) вычислительные затраты
(время выполнения, устойчивость результата при повторных запусках генетического алгоритма).

\subsection{Метрики сравнения}
По результатам работы формируются выходные файлы в формате XLSX/CSV,
содержащие распределение заказов по оборудованию, раскройные комбинации,
число ножей и потери ширины, а также вспомогательные показатели.
Для сопоставления решений используются следующие метрики:
\begin{enumerate}
    \item \textbf{Потери материала} (потери ширины) — суммарная невостребованная часть ширины при раскрое.
    \item \textbf{Число переналадок ножей} — показатель сложности реализации плана на ПРС,
    отражающий число изменений конфигурации ножей между последовательными раскроями.
    \item \textbf{Время вычисления} — время работы алгоритма до получения результата.
    \item \textbf{Стабильность} (для генетического алгоритма) — разброс качества решений при повторных запусках
    с различными начальными условиями при фиксированных параметрах популяции.
\end{enumerate}

\section{Результаты экспериментов}
\label{sec:results}

Результаты работы программного комплекса формируются в виде выходного файла (XLSX/CSV),
который включает:
\begin{itemize}
    \item распределение заказов по бумагоделательным машинам и продольно-резательным станкам;
    \item наборы раскройных комбинаций (состав раскроя и количество ножей);
    \item потери ширины по каждой операции раскроя и агрегированные показатели по плану;
    \item статистические показатели, включая число переналадок ножей и итоговую «эффективность» решения.
\end{itemize}

При проведении экспериментов рекомендуется представлять результаты в табличной и графической форме:
\begin{itemize}
    \item таблицы качества решений (потери, переналадки) для каждого набора данных и каждого режима;
    \item графики зависимости качества решения и времени расчёта от числа заказов и числа машин;
    \item для задач малой размерности — сравнение результата генетического алгоритма с оптимальным решением,
    найденным полным перебором (в абсолютных значениях и/или в относительном отклонении).
\end{itemize}

\section{Анализ сильных и слабых сторон}
\label{sec:strengths_weaknesses}

\subsection{Оптимальные условия применения}
Разработанный программный комплекс целесообразно использовать в двух типовых сценариях:
\begin{enumerate}
    \item \textbf{Задачи малой размерности} — применение режима полного перебора обеспечивает получение оптимальных решений,
    что полезно для точного планирования небольших партий и для тестирования/верификации.
    \item \textbf{Задачи средней и большой размерности} — применение генетического алгоритма позволяет получать
    качественные решения за разумное время, что соответствует практическим требованиям планирования на предприятии.
\end{enumerate}

К дополнительным практическим преимуществам относятся поддержка распространённых форматов данных,
наличие графического интерфейса для технологов и инженеров, а также модульная архитектура,
позволяющая расширять функциональность программного комплекса.

\subsection{Ограничения и узкие места}
Основные ограничения связаны со свойствами используемых режимов оптимизации:
\begin{itemize}
    \item \textbf{Полный перебор} обладает экспоненциальной трудоёмкостью и применим только для задач малой размерности.
    \item \textbf{Генетический алгоритм} является стохастическим методом, поэтому качество решения может зависеть от
    параметров популяции и начальных условий; для повышения устойчивости могут потребоваться повторные запуски и настройка параметров.
\end{itemize}

Кроме того, качество результатов существенно зависит от полноты и корректности исходных данных,
в частности от корректного задания ограничений оборудования и параметров заказов.

\section{Пример практического применения}
\label{sec:practical_example}

Практическое применение программного комплекса соответствует следующему регламенту работы:
\begin{enumerate}
    \item Пользователь подготавливает входной файл заказов в формате CSV или XLSX,
    где каждая строка соответствует заказу и содержит требуемые параметры (размеры, объёмы, марка, сроки и др.).
    \item В графическом интерфейсе выбирается входной файл и задаётся путь для сохранения результата.
    \item Выбирается режим оптимизации:
    \begin{itemize}
        \item полный перебор — для небольших задач и проверки оптимальности;
        \item генетический поиск — для задач большей размерности.
    \end{itemize}
    \item При выборе генетического поиска задаются параметры (например, размер популяции),
    после чего выполняется запуск расчёта.
    \item По завершении формируется выходной файл в формате XLSX/CSV,
    содержащий распределение заказов по машинам, раскройные комбинации,
    число ножей, потери ширины и статистические показатели, включая число переналадок.
\end{enumerate}

Пример экранных форм интерфейса, а также примеры входного файла и выходного решения
целесообразно привести в приложениях к диссертации.

\section{Выводы по главе}
\label{sec:chapter3_conclusions}

В главе представлено описание программной реализации разработанного подхода и методики экспериментального исследования.
Показано, что программный комплекс поддерживает обработку входных данных в форматах CSV/XLSX,
формирование результатов в форматах XLSX/CSV со статистическими показателями, а также два режима оптимизационного поиска:
генетический алгоритм и полный перебор. Использование графического интерфейса на основе Tkinter и запуск вычислений
в отдельном потоке обеспечивают удобство применения программного комплекса в инженерной практике.

Предложенная методика экспериментов позволяет оценить качество решений по показателям потерь материала и числа переналадок,
а также определить границы применимости точного режима полного перебора и приближённого режима генетического поиска.

% Codex

\section{Введение}\label{sec:software/intro}
В данной главе описывается программная реализация предложенного в работе алгоритма
генетического поиска для задачи линейного раскроя в контексте целлюлозно-бумажного
производства. Цель главы --- обеспечить полноту и воспроизводимость: показать, каким
образом математическая модель и алгоритмические решения трансформированы в
программный код, как организованы модули, какие библиотеки использованы, как
проверялась корректность результатов и каким образом пользователь взаимодействует
с системой.

Структура главы включает архитектурное описание в терминах C4, рассмотрение
моделей данных, описание алгоритмической реализации, форматов ввода/вывода,
особенностей запуска (CLI и GUI), используемых библиотек, тестирования и
воспроизводимости экспериментов. Такой подход позволяет связать формальную модель
с практической реализацией, а также обеспечивает прозрачность для дальнейшего
развития программного комплекса и внедрения на производственных предприятиях.

\section{Архитектура в рамках C4}\label{sec:software/c4}
\subsection{Контекст системы (C4-1)}
\begin{figure}[htbp]
    \centering
    \begin{tikzpicture}[
        node distance=12mm and 18mm,
        box/.style={draw, rounded corners, align=center, minimum width=3.6cm, minimum height=1.2cm},
        actor/.style={draw, rounded corners, align=center, minimum width=3.2cm, minimum height=1.2cm, fill=gray!10},
        arrow/.style={-Latex, thick}
    ]
        \node[box, fill=blue!10] (system) {\textbf{Система оптимизации раскроя}\\\footnotesize{Python-приложение}};
        \node[actor, left=of system] (user) {\textbf{Инженер-планировщик}\\\footnotesize{задает входные данные}};
        \node[actor, right=of system] (files) {\textbf{Файловая система}\\\footnotesize{CSV/Excel}};
        \draw[arrow] (user) -- node[above]{\footnotesize параметры и команды} (system);
        \draw[arrow] (system) -- node[above]{\footnotesize отчеты и планы} (files);
        \draw[arrow] (files) -- node[below]{\footnotesize входные данные} (system);
    \end{tikzpicture}
    \caption{Контекст системы: акторы и внешние связи}
    \label{fig:c4-context}
\end{figure}

\subsection{Контейнеры (C4-2)}
\begin{figure}[htbp]
    \centering
    \begin{tikzpicture}[
        node distance=10mm and 18mm,
        box/.style={draw, rounded corners, align=center, minimum width=3.2cm, minimum height=1.2cm},
        arrow/.style={-Latex, thick}
    ]
        \node[box, fill=blue!10] (cli) {\textbf{CLI}\\\footnotesize main.py};
        \node[box, fill=blue!10, right=of cli] (gui) {\textbf{GUI}\\\footnotesize gui.py};
        \node[box, fill=green!10, below=of cli] (algo) {\textbf{Алгоритмы}\\\footnotesize algo/};
        \node[box, fill=green!10, right=of algo] (models) {\textbf{Модели}\\\footnotesize models/};
        \node[box, fill=yellow!15, below=of algo] (io) {\textbf{Ввод/вывод}\\\footnotesize inoutput/};

        \draw[arrow] (cli) -- (algo);
        \draw[arrow] (gui) -- (algo);
        \draw[arrow] (algo) -- (models);
        \draw[arrow] (algo) -- (io);
        \draw[arrow] (io) -- (models);
    \end{tikzpicture}
    \caption{Контейнеры приложения и основные зависимости}
    \label{fig:c4-container}
\end{figure}

\subsection{Компоненты (C4-3)}
\begin{figure}[htbp]
    \centering
    \begin{tikzpicture}[
        node distance=8mm and 12mm,
        comp/.style={draw, rounded corners, align=center, minimum width=3.2cm, minimum height=0.9cm},
        arrow/.style={-Latex, thick}
    ]
        \node[comp, fill=green!10] (genetic) {GeneticSearch};
        \node[comp, fill=green!10, right=of genetic] (mutation) {GlobalMutate};
        \node[comp, fill=green!10, right=of mutation] (cross) {Crossingover};
        \node[comp, fill=green!10, below=of genetic] (enum) {FullEnumeration};
        \node[comp, fill=green!10, right=of enum] (builder) {CuttingPlansBuilder};
        \node[comp, fill=green!10, right=of builder] (queue) {EventQueueBuilder};

        \node[comp, fill=blue!10, below=of enum] (globalplan) {GlobalPlan};
        \node[comp, fill=blue!10, right=of globalplan] (cuttingplan) {CuttingPlan / Layout};
        \node[comp, fill=blue!10, right=of cuttingplan] (makingplan) {MakingPlan};
        \node[comp, fill=blue!10, below=of globalplan] (problem) {Problem};
        \node[comp, fill=blue!10, right=of problem] (rolls) {Rolls / Orders};
        \node[comp, fill=blue!10, right=of rolls] (machines) {Machines / Storage};

        \draw[arrow] (genetic) -- (mutation);
        \draw[arrow] (genetic) -- (cross);
        \draw[arrow] (enum) -- (builder);
        \draw[arrow] (genetic) -- (builder);
        \draw[arrow] (builder) -- (globalplan);
        \draw[arrow] (queue) -- (globalplan);

        \draw[arrow] (globalplan) -- (cuttingplan);
        \draw[arrow] (globalplan) -- (makingplan);
        \draw[arrow] (problem) -- (globalplan);
        \draw[arrow] (rolls) -- (cuttingplan);
        \draw[arrow] (machines) -- (cuttingplan);
    \end{tikzpicture}
    \caption{Компоненты алгоритмического ядра и модельного слоя}
    \label{fig:c4-component}
\end{figure}
\subsection{Контекст (System Context)}\label{subsec:software/c4/context}
Программная система предназначена для автоматизированного построения производственных
планов раскроя и выпуска рулонов бумаги. Пользователь (инженер планирования)
подаёт входные данные (CSV/Excel), выбирает метод решения и получает выходной отчет
в формате Excel/CSV, содержащий планы раскроя, расписание и метрики качества.
Система функционирует как автономное приложение, не требующее внешнего сервера.

\paragraph{Акторы:}
\begin{itemize}
    \item инженер-технолог или планировщик производства;
    \item файловая система как источник и приемник данных;
    \item внешние библиотеки (Pandas/OpenPyXL) как инфраструктура чтения/записи.
\end{itemize}

\subsection{Контейнеры (Container Diagram)}\label{subsec:software/c4/container}
Архитектура реализована в виде нескольких логических контейнеров внутри одного
Python-приложения:

\begin{enumerate}
    \item \textbf{CLI-приложение} --- модуль командной строки; обеспечивает запуск
    расчёта с аргументами, чтение файлов и сохранение результатов.
    \item \textbf{GUI-приложение} --- графический интерфейс (Tkinter), позволяющий
    выбрать файлы и параметры запуска, а также визуально контролировать выполнение.
    \item \textbf{Алгоритмический модуль} --- содержит реализацию генетического поиска,
    полного перебора и построителей планов.
    \item \textbf{Модельный слой} --- набор классов предметной области (заказы,
    машины, планы, события, склад).
    \item \textbf{IO-модуль} --- парсинг входных данных и генерация отчетов.
\end{enumerate}

\subsection{Компоненты (Component Diagram)}\label{subsec:software/c4/component}
Внутри контейнеров выделяются ключевые компоненты:

\paragraph{Алгоритмический модуль:}
\begin{itemize}
    \item \textit{GeneticSearch} --- основной оптимизационный алгоритм;
    \item \textit{GlobalMutate} --- мутация глобального плана;
    \item \textit{Crossingover} --- отбор лучшего плана;
    \item \textit{FullEnumeration} --- точный перебор (для малых задач);
    \item \textit{PlanBuilder} --- конструктор планов раскроя;
    \item \textit{EventQueueBuilder} --- построение очереди событий при наличии
    временных ограничений.
\end{itemize}

\paragraph{Модельный слой:}
\begin{itemize}
    \item \textit{Problem} --- постановка задачи;
    \item \textit{GlobalPlan} --- глобальный план производства и раскроя;
    \item \textit{CuttingPlan} и \textit{CuttingLayout} --- структура раскроя;
    \item \textit{MakingPlan} --- план производства тамбуров;
    \item \textit{Order/Roll} --- заказы и типы рулонов;
    \item \textit{Storage/StorageState} --- склад и его состояние;
    \item \textit{EventQueue} --- очередь событий исполнения планов.
\end{itemize}
\begin{table}[htbp]
    \centering
    \begin{threeparttable}
        \caption{Сопоставление сущностей математической модели и классов программной реализации}
        \label{tab:model_mapping}
        \begin{tabular}{|p{3.4cm}|p{3.2cm}|p{4.2cm}|p{4.2cm}|}
            \hline
            \textbf{Сущность мат. модели} & \textbf{Символ} & \textbf{Класс/модуль} & \textbf{Комментарий} \\
            \hline
            Заказы & $O=\{o_i\}$ & \texttt{RollOrder} (models/order.py) & Экземпляр заказа с шириной, количеством, плотностью, маркой, слоями и дедлайном.\\
            \hline
            БДМ & $M=\{m_i\}$ & \texttt{PaperMakingMachine} (models/pm\_machine.py) & Параметры БДМ: ширина, скорость, смена плотности.\\
            \hline
            ПРС/БРС & $C=\{c_i\}$ & \texttt{SlittingMachine} (models/slitting\_machine.py) & Параметры раскроя: ширина, ножи, скорость, смена ножей, тип (small/normal).\\
            \hline
            Склады & $T=\{t_i\}$ & \texttt{Storage}, \texttt{StorageState} (models/storage.py) & Складские ограничения по ширине и вместимости, состояние запасов.\\
            \hline
            Тамбуры/рулоны & $X_{i,j}$, $y_{i,j,k}$ & \texttt{MasterRoll}, \texttt{SlittedRoll}, \texttt{SlittedMasterRoll} (models/roll.py) & Представление тамбуров и конечных рулонов; тип определяет, где используется.\\
            \hline
            План производства БДМ & $\overline{\overline{X}}$ & \texttt{MakingPlan} (models/making\_plan.py) & Последовательность тамбуров для БДМ.\\
            \hline
            План раскроя ПРС & $\overline{\overline{Y}}$ & \texttt{CuttingPlan} (models/cutting\_plan.py) & Содержит набор раскроев (layouts) для станка.\\
            \hline
            Раскрой & $Y_{i,j}$ & \texttt{CuttingLayout} (models/cutting\_layout.py) & Конкретный набор фомтов в одном раскрое.\\
            \hline
            Очередь событий & $\overline{\overline{Z}}$ & \texttt{GlobalPlanEvent} + наследники (models/event\_queue.py) & Расписание событий раскроя/производства.\\
            \hline
            Глобальный план & $G=(\overline{\overline{X}},\overline{\overline{Y}},\overline{\overline{Z}})$ & \texttt{GlobalPlan} (models/global\_plan.py) & Центральный объект решения: планы раскроя, производства, события.\\
            \hline
            Постановка задачи & --- & \texttt{Problem} (models/problem.py) & Агрегирует заказы, машины, ограничения по времени и складам.\\
            \hline
        \end{tabular}
    \end{threeparttable}
\end{table}

\subsection{Уровень кода (Code Level)}\label{subsec:software/c4/code}
На уровне кода архитектура реализована через классы и функции, сгруппированные по
папкам. Такое разделение отражает принципы модульности и облегчает расширение:

\begin{itemize}
    \item \textbf{models/} --- классы предметной области;
    \item \textbf{algo/} --- алгоритмы оптимизации и построения планов;
    \item \textbf{inoutput/} --- чтение входных данных и генерация отчетов;
    \item \textbf{main.py, gui.py} --- пользовательские точки входа.
\end{itemize}

\section{Модели данных и предметная область}\label{sec:software/models}

\subsection{Сущности заказа и продукции}\label{subsec:software/models/orders}
Центральная единица задачи --- \textit{заказ} на изготовление рулонов бумаги.
Заказ описывается параметрами: ширина, количество рулонов, марка бумаги, плотность,
число слоев и срок выполнения. В программной модели заказ представлен классом
\texttt{RollOrder}. Он наследует общий класс \texttt{SlittedRoll} и дополняет его
параметрами количества и дедлайна.

Сами рулоны описываются через иерархию классов:
\begin{itemize}
    \item \texttt{Roll} --- базовый тип рулона;
    \item \texttt{MasterRoll} --- тамбур (широкий рулон, производимый БДМ);
    \item \texttt{SlittedRoll} --- конечный рулон после нарезки;
    \item \texttt{SlittedMasterRoll} --- рулон, предназначенный для последующей
    нарезки на БРС.
\end{itemize}

\subsection{Оборудование}\label{subsec:software/models/machines}
В модели учитываются два типа оборудования:
\begin{itemize}
    \item \textbf{Бумагоделательные машины (БДМ)} --- задаются шириной производимого
    тамбура, скоростью производства и временем смены плотности.
    \item \textbf{Продольно-резательные станки (ПРС/БРС)} --- задаются шириной,
    числом ножей, скоростью раскроя, временем перестановки ножей, типом (обычный/малый).
\end{itemize}

\subsection{Планы производства}\label{subsec:software/models/plans}
Структура плана соответствует математической модели:

\begin{enumerate}
    \item \textbf{CuttingLayout} --- один раскрой (набор форматов).
    \item \textbf{CuttingPlan} --- последовательность раскроев для одного станка.
    \item \textbf{MakingPlan} --- последовательность тамбуров для одной БДМ.
    \item \textbf{GlobalPlan} --- объединяет планы по всем БДМ и ПРС и содержит очередь
    событий, отражающую календарное выполнение.
\end{enumerate}

\subsection{Очередь событий и склад}\label{subsec:software/models/events}
Если в задаче заданы сроки выполнения заказов, требуется моделировать временной
аспект производства. В этой части используются события:
\begin{itemize}
    \item \textit{GlobalPlanSlittingMachineEvent} --- событие раскроя;
    \item \textit{GlobalPlanPaperMakingMachineEvent} --- событие выпуска тамбура.
\end{itemize}

Склад моделируется через классы \texttt{Storage} и \texttt{StorageState}. Хранится
информация о доступных рулонах, соблюдаются ограничения по вместимости и ширине.

\section{Алгоритмическая реализация}\label{sec:software/algorithms}

\subsection{Генетический поиск}\label{subsec:software/algorithms/genetic}
Алгоритм реализует модель, описанную в главе с математической постановкой. Основные
этапы:

\begin{enumerate}
    \item \textbf{Инициализация} --- создаются случайные планы, удовлетворяющие
    ограничениям.
    \item \textbf{Мутация} --- перестановка фоматов внутри раскроев и самих раскроев.
    \item \textbf{GlobalMutate} --- формирование нового глобального плана, пересчет
    очереди событий, проверка улучшения целевой функции.
    \item \textbf{Crossingover} --- отбор лучшего из пары.
\end{enumerate}

Генетический поиск является стохастическим алгоритмом. Качество решения зависит
от размера начальной популяции и глубины итераций.

\subsection{Полный перебор}\label{subsec:software/algorithms/enumeration}
Для малых размерностей реализован режим полного перебора. Он строит планы рекурсивно,
добавляя заказы к планам раскроя и выбирая лучшую комбинацию. Этот режим служит
для верификации алгоритма и оценки качества приближенных методов.

\subsection{Планирование времени}\label{subsec:software/algorithms/schedule}
Если заданы дедлайны, используется генерация очереди событий. Алгоритм моделирует
параллельную работу БДМ и ПРС, учитывая состояние склада. Очередь событий строится
как последовательность операций с учетом времени выполнения и возможных конфликтов.

\subsection{Оценка качества решения}\label{subsec:software/algorithms/quality}
Критерии соответствуют целевой функции:
\begin{enumerate}
    \item суммарная ширина отходов;
    \item максимальное число раскроев для станка;
    \item количество перестановок ножей.
\end{enumerate}
В коде критерии реализованы в виде лексикографической функции сравнения планов.

\subsection{Сложность алгоритма}\label{subsec:software/algorithms/complexity}
Размер популяции уменьшается вдвое на каждой итерации, что даёт
$O(|S_0| \log |S_0|)$ итераций. Стоимость каждой итерации определяется числом форматов
(сумма заказов). Общая асимптотика соответствует оценкам в постановочной главе.

\section{Форматы ввода и вывода}\label{sec:software/io}

\subsection{Формат CSV}\label{subsec:software/io/csv}
CSV-формат используется для минимальных экспериментов и быстрого обмена данными.
Каждая строка содержит тип параметра (например, ширина станка или заказ),
его идентификатор и значение. Такой формат удобен для автоматической генерации
и обработки скриптами.

\subsection{Формат Excel}\label{subsec:software/io/excel}
Excel-формат применяется для практической работы. Используются листы:
\begin{itemize}
    \item \textit{Slitting Machines} --- параметры ПРС/БРС;
    \item \textit{Paper Making Machines} --- параметры БДМ;
    \item \textit{Orders} --- перечень заказов;
    \item \textit{Conditions} --- дополнительные параметры (кромка, старт, марки);
    \item \textit{Storages} --- склады и ограничения.
\end{itemize}

\subsection{Выходной отчет}\label{subsec:software/io/output}
Система формирует выходные отчеты в CSV/Excel. При наличии дедлайнов добавляется
лист расписания, содержащий последовательность событий, использованные тамбуры,
перестановки ножей и суммарные метрики.

\section{Пользовательские сценарии}\label{sec:software/usage}

\subsection{CLI-режим}\label{subsec:software/usage/cli}
CLI позволяет запускать расчет с параметрами:
\begin{verbatim}
python main.py input.xlsx output.xlsx --method genetic_search
\end{verbatim}
Параметры позволяют выбрать формат ввода, формат вывода и размер популяции.

\subsection{GUI-режим}\label{subsec:software/usage/gui}
GUI реализован с использованием Tkinter. Пользователь выбирает входной и выходной
файл, метод решения и размер популяции. Выполнение происходит в отдельном потоке,
что позволяет избежать блокировки интерфейса.

\section{Используемые библиотеки}\label{sec:software/libs}
Реализация опирается на стандартную библиотеку Python и несколько основных пакетов:
\begin{itemize}
    \item \textbf{Pandas} --- обработка табличных данных;
    \item \textbf{OpenPyXL} --- запись Excel-отчетов;
    \item \textbf{Tkinter} --- GUI;
    \item \textbf{argparse} --- обработка аргументов CLI;
    \item \textbf{datetime} --- работа со временем.
\end{itemize}

Выбор таких библиотек обусловлен их стабильностью, распространенностью в инженерных
задачах и поддержкой кроссплатформенной разработки.

\section{Тестирование и валидация}\label{sec:software/tests}

\subsection{Уровни тестирования}\label{subsec:software/tests/levels}
Тестирование включает:
\begin{enumerate}
    \item модульные тесты --- корректность отдельных классов и функций;
    \item интеграционные тесты --- полный цикл ``ввод $\rightarrow$ расчет $\rightarrow$ вывод'';
    \item экспериментальная валидация --- сравнение с эталонными результатами.
\end{enumerate}

\subsection{Сценарии тестирования}\label{subsec:software/tests/scenarios}
Были использованы следующие сценарии:
\begin{itemize}
    \item отсутствие дедлайнов;
    \item наличие дедлайнов и складов;
    \item наличие заказов с малыми ширинами (БРС);
    \item разный размер популяции;
    \item сравнение с полным перебором на малых данных.
\end{itemize}

\subsection{Метрики корректности}\label{subsec:software/tests/metrics}
Проверка решения включает:
\begin{itemize}
    \item соблюдение ограничений ширины и ножей;
    \item выполнение всех заказов;
    \item отсутствие превышения склада;
    \item корректность дат завершения заказов при дедлайнах.
\end{itemize}

\section{Эксперименты и воспроизводимость}\label{sec:software/experiments}

\subsection{Конфигурации экспериментов}\label{subsec:software/experiments/config}
В экспериментах использовались данные, аналогичные приведённым в статье: несколько
БДМ и ПРС, набор заказов с разной шириной и плотностью, склады.

\subsection{Измеряемые показатели}\label{subsec:software/experiments/metrics}
Для сравнения конфигураций анализируются:
\begin{enumerate}
    \item отходы (суммарная ширина);
    \item число раскроев;
    \item перестановки ножей;
    \item время выполнения.
\end{enumerate}

\subsection{Воспроизводимость}\label{subsec:software/experiments/repro}
Так как алгоритм стохастический, рекомендуется фиксировать seed генератора случайных
чисел или выполнять серию запусков с последующим усреднением результатов. В
практическом применении допустима небольшая вариация результатов, так как отбор
планов всегда строится по улучшению целевой функции.

\section{Заключение}\label{sec:software/conclusion}
В данной главе подробно описана программная реализация алгоритма генетического поиска
для задачи линейного раскроя с учетом требований целлюлозно-бумажного производства.
Архитектура построена модульно и включает отдельные контейнеры для алгоритмов,
моделей данных и ввода/вывода. Реализация обеспечивает как командный режим
использования, так и графический интерфейс, что делает систему удобной для
промышленного применения.

Система покрывает основные элементы математической модели: ограничения по ширине и
ножам, работу складов, учет плотности и марок, сроки выполнения заказов, формирование
очереди событий. Проведенное тестирование подтверждает корректность реализации, а
эксперименты демонстрируют сопоставимость результатов с теоретическими ожиданиями.

Таким образом, программный комплекс представляет собой практическую основу для
внедрения методов интеллектуального планирования в промышленности и может служить
платформой для дальнейших исследований и расширений.

\section{Тестирование программной реализации}
\label{sec:software-testing}

В данном разделе приведено описание всех тестов, включенных в проект. Тесты
сгруппированы по пакетам и соответствуют уровням проверки: модульные тесты
модельного слоя, модульные тесты алгоритмов, тесты ввода/вывода и интеграционные
(сквозные) тесты полного запуска решателя.

\subsection{Тесты алгоритмического модуля (\texttt{algo/tests})}
\label{subsec:software-testing/algo}

\paragraph{test\_crossingover.py}
\begin{itemize}
    \item \textbf{test\_crossingover} --- проверяет, что оператор кроссинговера
    возвращает один из двух входных глобальных планов и не ухудшает их метрики
    (отходы и перестановки ножей).
\end{itemize}

\paragraph{test\_event\_queue\_builder.py}
\begin{itemize}
    \item \textbf{test\_total\_speed\_global\_plan} --- проверяет, что общее время
    выполнения глобального плана совпадает с максимальным временем раскроя на
    отдельных станках.
    \item \textbf{test\_global\_plan\_time\_limit} --- проверяет корректность
    обработки дедлайнов: при слишком жестких сроках план становится недопустимым.
    \item \textbf{test\_pms\_global\_plan\_time\_limit} --- аналогичная проверка,
    но с учетом бумагоделательных машин и сформированных производственных планов.
    \item \textbf{test\_global\_plan\_with\_storage\_capacity} --- проверяет, что
    расписание учитывает ограничения склада и включает моменты ожидания, если
    нет доступных рулонов.
\end{itemize}

\paragraph{test\_full\_enumeration.py}
\begin{itemize}
    \item \textbf{test\_full\_enumeration} --- проверяет, что полный перебор
    возвращает корректное решение, не хуже случайных планов.
    \item \textbf{test\_article\_best\_solution} --- проверяет соответствие
    результату для примера из статьи: найденное решение не хуже эталонного.
\end{itemize}

\paragraph{test\_genetic\_search.py}
\begin{itemize}
    \item \textbf{test\_genetic\_search} --- проверяет, что генетический поиск
    возвращает корректный план и улучшает качество по сравнению со случайными
    решениями.
\end{itemize}

\paragraph{test\_global\_mutate.py}
\begin{itemize}
    \item \textbf{test\_global\_mutate} --- проверяет, что глобальная мутация
    не ухудшает решение и иногда улучшает его.
    \item \textbf{test\_event\_rebuild} --- проверяет, что после мутации очередь
    событий пересчитывается (события могут отличаться по времени или типу).
\end{itemize}

\paragraph{test\_mutate.py}
\begin{itemize}
    \item \textbf{test\_mutate} --- проверяет, что локальная мутация плана
    раскроя действительно меняет конфигурацию (перестановка форматов/раскроев).
\end{itemize}

\paragraph{test\_pmm\_plans\_builder.py}
\begin{itemize}
    \item \textbf{test\_generate\_making\_plans\_simple\_case} --- проверяет
    корректное формирование плана производства тамбуров для одной БДМ.
    \item \textbf{test\_generate\_making\_plans\_multiple\_identical\_machines} ---
    проверяет распределение тамбуров между несколькими одинаковыми БДМ.
    \item \textbf{test\_generate\_making\_plans\_different\_widths} --- проверяет
    соответствие ширин тамбуров и БДМ при нескольких типах машин.
    \item \textbf{test\_total\_time\_pms\_global\_plan} --- проверяет корректность
    вычисления суммарного времени плана при наличии БДМ.
\end{itemize}

\paragraph{test\_random\_global\_plan.py}
\begin{itemize}
    \item \textbf{test\_randomness} --- проверяет, что генератор случайных планов
    формирует различающиеся решения.
    \item \textbf{test\_constraints} --- проверяет выполнение ограничений в
    случайных планах.
    \item \textbf{test\_constraints\_strictly} --- аналогичная проверка при
    строгой конфигурации ширин станков.
    \item \textbf{test\_invalid} --- проверяет, что при невозможности решения
    генерация плана вызывает исключение.
    \item \textbf{test\_time\_pms} --- проверяет корректность планов при наличии
    БДМ и временных ограничений.
    \item \textbf{test\_density} --- проверяет, что в раскрое сохраняется
    однородность плотности рулонов.
\end{itemize}

\subsection{Тесты модельного слоя (\texttt{models/tests})}
\label{subsec:software-testing/models}

\paragraph{test\_cutting\_layout.py}
\begin{itemize}
    \item \textbf{test\_cutting\_layout} --- проверяет валидность раскроя при
    разных ширинах и числе ножей, а также вычисление отходов.
    \item \textbf{test\_format\_dicts} --- проверяет корректность словаря форматов
    (количество рулонов одного типа в раскрое).
\end{itemize}

\paragraph{test\_cutting\_plan.py}
\begin{itemize}
    \item \textbf{test\_cutting\_plan} --- проверяет валидность плана и расчет
    отходов и перестановок ножей.
    \item \textbf{test\_formats\_dict} --- проверяет распределение заказов по плану.
    \item \textbf{test\_total\_speed\_cutting\_plan} --- проверяет расчет времени
    раскроя с учетом скорости производства и перестановок.
    \item \textbf{test\_total\_speed\_layered\_cutting\_plan} --- проверяет
    корректность времени при многослойных заказах.
\end{itemize}

\paragraph{test\_global\_plan.py}
\begin{itemize}
    \item \textbf{test\_formats\_dict} --- проверяет корректность агрегирования
    количества заказов по всему глобальному плану.
    \item \textbf{test\_good\_global\_plan} --- проверяет корректность базового
    глобального плана, его отходов и перестановок.
    \item \textbf{test\_article\_global\_plan} --- проверяет пример плана из
    статьи (ожидаемые метрики).
    \item \textbf{test\_article\_solution\_global\_plan} --- проверяет
    эталонное решение из статьи.
    \item \textbf{test\_article\_best\_global\_plan} --- проверяет план с
    нулевыми отходами и минимальными перестановками.
    \item \textbf{test\_global\_plan\_comparison} --- проверяет корректность
    сравнения планов по критериям.
    \item \textbf{test\_max\_slitting\_machine\_plans} --- проверяет расчет
    максимального числа раскроев на станке.
\end{itemize}

\paragraph{test\_making\_plan.py}
\begin{itemize}
    \item \textbf{test\_making\_plan\_total\_time} --- проверяет расчет времени
    производства тамбуров.
    \item \textbf{test\_making\_plan\_equality} --- проверяет сравнение планов
    по количеству тамбуров.
    \item \textbf{test\_making\_plan\_string\_representation} --- проверяет
    строковое представление плана.
\end{itemize}

\paragraph{test\_problem.py}
\begin{itemize}
    \item \textbf{test\_one\_plus\_one} --- проверка корректности тестовой
    инфраструктуры.
    \item \textbf{test\_problem} --- проверяет базовую корректность объекта
    задачи (число заказов и станков).
    \item \textbf{test\_problem\_with\_knives\_and\_speed} --- проверяет загрузку
    параметров станков (ножи и скорости).
\end{itemize}

\paragraph{test\_roll.py}
\begin{itemize}
    \item \textbf{test\_is\_compatible} --- проверяет совместимость рулонов по
    количеству слоев.
\end{itemize}

\subsection{Тесты ввода/вывода (\texttt{inoutput/tests})}
\label{subsec:software-testing/inoutput}

\paragraph{test\_input.py}
\begin{itemize}
    \item \textbf{test\_parse\_input} --- проверяет базовый парсинг CSV.
    \item \textbf{test\_parse\_input\_with\_knives} --- проверяет парсинг ножей.
    \item \textbf{test\_parse\_input\_with\_additional\_attributes} --- проверяет
    чтение скоростей производства и перестановки ножей.
    \item \textbf{test\_parse\_input\_with\_time\_limit} --- проверяет чтение
    дедлайнов и времени запуска.
    \item \textbf{test\_parse\_input\_with\_paper\_making\_machines} --- проверяет
    парсинг БДМ и их параметров.
    \item \textbf{test\_storage\_capacity\_input} --- проверяет парсинг складов.
    \item \textbf{test\_parse\_orders\_with\_density} --- проверяет чтение плотности.
    \item \textbf{test\_parse\_orders\_with\_density\_change\_speed} --- проверяет
    чтение скорости смены плотности на БДМ.
    \item \textbf{test\_parse\_brand} --- проверяет чтение марки бумаги.
    \item \textbf{test\_parse\_brand\_ordered} --- проверяет чтение порядка марок
    и времени переналадки.
    \item \textbf{test\_excel\_input} --- проверяет чтение Excel со всеми
    параметрами (станки, заказы, склад, марки).
    \item \textbf{test\_excel\_simple\_input} --- проверяет упрощенный Excel.
    \item \textbf{test\_layers\_input} --- проверяет чтение числа слоев.
    \item \textbf{test\_edge\_input} --- проверяет чтение кромки.
    \item \textbf{test\_small\_order\_input} --- проверяет чтение типа заказа
    (ПРС/БРС).
\end{itemize}

\paragraph{test\_output.py}
\begin{itemize}
    \item \textbf{test\_output\_solution} --- проверяет корректность CSV-вывода
    (сравнение с эталонным файлом).
    \item \textbf{test\_output\_solution\_excel} --- проверяет генерацию Excel
    отчета и его наличие.
\end{itemize}

\subsection{Интеграционные тесты (\texttt{tests/})}
\label{subsec:software-testing/integration}

\paragraph{test\_functional.py}
\begin{itemize}
    \item \textbf{test\_pmm} --- запуск полного цикла расчета с БДМ.
    \item \textbf{test\_pmm\_with\_time} --- запуск полного цикла с учетом времени.
    \item \textbf{test\_storage} --- проверка корректного ожидания при складских
    ограничениях.
    \item \textbf{test\_density} --- проверка согласованности плотности в планах.
    \item \textbf{test\_density\_with\_time} --- запуск задачи с плотностью и
    дедлайнами.
    \item \textbf{test\_brand} --- проверка согласованности марок.
    \item \textbf{test\_brand\_ordered} --- проверка соблюдения порядка марок.
    \item \textbf{test\_density\_brand} --- проверка совмещения плотности и марок.
    \item \textbf{test\_layers} --- запуск задачи со слоями.
    \item \textbf{test\_layers\_with\_time} --- слои с временными ограничениями.
    \item \textbf{test\_subslitting} --- проверка сценария БРС.
    \item \textbf{test\_subslitting\_with\_time} --- БРС с дедлайнами.
    \item \textbf{test\_edge} --- проверка корректной обработки кромки.
    \item \textbf{test\_full\_problem} --- полный интеграционный сценарий
    со всеми ограничениями.
\end{itemize}

\paragraph{test\_benchmark.py}
\begin{itemize}
    \item \textbf{test\_benchmark\_1\_full\_enumeration} --- бенчмарк с полным
    перебором (отключен по умолчанию).
    \item \textbf{test\_benchmark\_1\_genetic\_search} --- бенчмарк генетического
    поиска (отключен по умолчанию).
    \item \textbf{test\_benchmark\_2\_full\_enumeration} --- второй сценарий
    бенчмарка (отключен).
    \item \textbf{test\_benchmark\_2\_genetic\_search} --- второй сценарий для GA
    (отключен).
    \item \textbf{test\_benchmark\_3\_full\_enumeration} --- третий сценарий
    бенчмарка (отключен).
    \item \textbf{test\_benchmark\_3\_genetic\_search} --- третий сценарий для GA
    (отключен).
    \item \textbf{test\_benchmark\_4\_full\_enumeration} --- четвертый сценарий
    бенчмарка (отключен).
    \item \textbf{test\_benchmark\_4\_genetic\_search} --- четвертый сценарий для
    GA (отключен).
\end{itemize}