\chapter{Теоретическая часть}\label{ch:theor_ch}
В данной главе формализуется рассматриваемая производственная задача и обосновывается выбор математической модели. Сначала описывается технологический контекст целлюлозно-бумажного производства и формулируются ключевые требования к плану выпуска и нарезки. Далее приводится обзор литературы по классической задаче линейного раскроя и её отраслевым обобщениям, включая постановки с переналадками, календарным планированием и складскими ограничениями. Завершает главу постановка задачи: вводятся обозначения, критерии оптимальности и набор ограничений, определяющих выполнимость глобального плана.

\section{Описание задачи}\label{sec:theor_ch/problem_description}
В разделе описывается производственная цепочка и фиксируются особенности постановки, которые должны быть отражены в математической модели. Для удобства восприятия сначала вводятся основные термины и элементы технологической схемы, затем формулируются критерии качества плана и перечень факторов, учитываемых в модели.

\subsection{Технологический контекст и элементы производственной цепочки}\label{subsec:theor_ch/problem_context}
Существует потребность в производстве рулонов бумаги, например на целлюлозно-бумажных комбинатах. Заказы по рулонам бумаги имеют множество параметров: ширину, количество рулонов, срок, марка бумаги, плотность и количество слоев.

Процесс производства заказов бумаги представляет собой сложную многостадийную технологическую цепочку, включающую подготовку целлюлозной массы, формирование и обработку бумажного полотна, а также его последующую нарезку на рулоны заданных форматов \cite{hubbe2021_energy}. Производство начинается с подачи целлюлозы и древесной массы в смесительные бассейны, где образуется однородная бумажная масса \cite{ippta_paper_and_water_2022}. Далее эта масса поступает на бумагоделательную машину (БДМ), где проходит последовательное обезвоживание, сушку и каландрирование, обеспечивая требуемую плотность и текстуру полотна. В завершении формируется тамбур — барабанный рулон для последующей переработки. БДМ может производить тамбуры только фиксированной длины, с заданной скоростью. На смену марки или плотности производимых тамбуров требуется время.

На следующем этапе заготовки поступают на продольно-резательные станки (ПРС), которые нарезают тамбур на рулоны нужных форматов \cite{sgp_ccp2021_cutting_stock_modes}. В случае, если ПРС ограничены рядом технологических параметров: максимальной шириной обрабатываемого тамбура \cite{minimizing_waste_multilevel_manufacturing_2025}, количеством доступных ножей \cite{cyberleninka_slitting_params} и временем, необходимым для их перестановки \cite{martina2022, pmc_kim_calendering_2022}. При нарезке тамбуров необходимо срезать края тамбуров. Такие края называются кромкой и они менее качественны чем бумага в центре. Это уменьшает доступную для нарезки ширину тамбура и требует установку ножей по краям на ПРС.

Некоторые заказы могут иметь больше одного слоя, тогда на ПРС подается столько же тамбуров, сколько и требуется слоев. БДМ может производить только однослойные тамбуры.

Также существуют бобино-резательные станки (БРС), которые имеют такие же параметры как и ПРС, но нарезают не тамбуры с БДМ, а рулоны, которые были нарезаны на других ПРС. БРС используются для нарезки заказов с небольшой шириной.

Если все ПРС заняты или требуется накопить несколько тамбуров для нарезки заказа, тамбуры и рулоны направляются на склад. Склады ограничены по вместимости и максимальной допустимой ширине рулонов. Потом со склада они отправляются на нарезку.

\subsection{Термины и определения}\label{subsec:theor_ch/terms}
Для дальнейшего изложения используются следующие термины.
\begin{description}
    \item[Заказ] Требование к выпуску заданного числа рулонов определённой ширины, марки и плотности с указанным сроком исполнения; в ряде случаев заказ является многослойным.
    \item[Тамбур] Рулон-полуфабрикат, выпускаемый на БДМ и используемый как исходный материал для продольной резки.
    \item[ПРС] Продольно-резательный станок, выполняющий резку тамбура на рулоны конечных форматов.
    \item[БРС] Бобино-резательный станок, выполняющий дополнительную резку уже нарезанных рулонов для узких форматов.
    \item[Кромка] Полоса по краям тамбура, подлежащая срезу и учитываемая как технологический отход.
    \item[Схема раскроя] Набор форматов, которые одновременно получаются из одного тамбура при одной операции резки.
    \item[Переналадка (перестановка ножей)] Изменение положения ножей на станке при переходе между схемами раскроя; приводит к простою и влияет на длительность производственного цикла.
    \item[Склад] Буфер хранения полуфабрикатов (тамбуров/рулонов) с ограничениями по вместимости и допустимой ширине.
    \item[Глобальный план] Согласованное решение по выпуску полуфабриката, операциям резки и их порядку во времени.
\end{description}

\subsection{Критерии качества и конфликт целей}\label{subsec:theor_ch/criteria}
На практике критерии качества плана противоречивы: снижение отхода часто требует более «тонкой» подгонки схем раскроя, что может увеличивать число раскроев и количество переналадок. В свою очередь, минимизация переналадок обычно достигается укрупнением партий и упрощением последовательности схем, что может приводить к росту отходов или увеличению времени ожидания заказов. Поэтому модель должна позволять сравнивать планы по нескольким показателям и выбирать компромиссное решение.

Так как нужно учитывать большое количество ограничений, то используются планы производства. Планы производства включают в себя планы производства бумаги на БДМ, планы нарезки на ПРС и БРС. Есть критерии, которые определяют эффективность плана производства:
\begin{enumerate}
    \item минимизацию отходов, что позволяет снизить издержки и уменьшить воздействие на окружающую среду \cite{haessler1991};
    \item минимизацию числа раскроев, так как каждый раскрой связан с дополнительными затратами времени и материальных ресурсов\cite{cerqueira2021, vanderbeck2000};
    \item минимизацию перестановок ножей, отражающих трудоёмкость переналадки и простой оборудования \cite{cyberleninka_slitting_params, mobasher2013}.
\end{enumerate}
Оптимизация именно этих критериев позволяет найти баланс между экономической эффективностью, качеством исполнения заказов и соблюдением технологических ограничений.

Особую сложность вызывает необходимость согласования планов работы БДМ и ПРС при наличии множества заказов, отличающихся по ширине, плотности, марке бумаги, срокам и дополнительным характеристикам \cite{keskinocak2002}. В модели вводится правило упорядочивания марок бумаги в производстве, так как смена марки связана с технологическим простоем и расходом сырья. Порядок марок может задаваться технологами до составления планов производства. Для складов задаются ограничения по вместимости: они отражают реальные лимиты на хранение полуфабрикатов, обеспечивая, чтобы план не приводил к переполнению и сбоям в логистике.

Таким образом, составление оптимального плана производства с учётом всех производственных параметров и ограничений является ключевым инструментом повышения эффективности работы предприятия. Это обеспечивает не только рост конкурентоспособности компании, но и вклад в устойчивое развитие отрасли за счёт ресурсосбережения и снижения негативного экологического воздействия. Математическая модель и алгоритм позволяют сформировать план производства и нарезки, минимизирующий отходы, переналадки и издержки, а также гарантировать своевременное выполнение заказов \cite{keskinocak2002}.

В математической модели нужно включить следующие факторы:
\begin{itemize}
    \item заказы представляют собой рулоны и имеют срок, ширину, количество рулонов, количество слоев, плотность, марку бумаги. В зависимости от ширины они должны быть выпущены на ПРС или БРС;
    \item марки бумаги должны выставляться на БДМ в соответствии с заданным порядком. Смена марки затрачивает время одинаковое для всех БДМ;
    \item для ПРС и БРС задается единый размер кромки тамбура. Половина этого размера должна быть срезана слева, а другая половина справа. При самой первой нарезке можно выставить нож вначале на отступ соответствующий половине размера кромки и не передвигать его при последующих нарезках. А другой край будет срезаться при нарезки рулонов, но важно чтобы нож не заходил на эту половину размера кромки в конце рулона;
    \item время начала производства;
    \item БДМ, каждый из которых имеет ширину выпускаемого тамбура, скорость производства бумаги, время смены плотности;
    \item ПРС и БРС, каждый из которых имеет максимальную ширину нарезаемого рулона, количество ножей, скорость нарезки одного раскроя с учетом количества слоев, скорость перестановки ножей;
    \item склады рулонов для нарезки, которые имеют максимальную вместимость по количеству рулонов и  максимально допустимую ширину рулона;
    \item момент появления тамбура на складе соответствует окончанию его производства на БДМ;
    \item при выполнении нарезки на ПРС или БРС со склада изымаются необходимые рулоны;
    \item БДМ и ПРС могут работать параллельно, если их процессы не конфликтуют по состоянию склада.
    \item итоговый план производства, который учитывает все ограничения и отражает последовательность действий для выполнения всех заказов.
\end{itemize}

\subsection{Принятые допущения и ограничения применимости}\label{subsec:theor_ch/assumptions}
При формализации задачи используются допущения, позволяющие построить модель, пригодную для алгоритмического поиска решений.
\begin{itemize}
    \item Модель ориентирована на планирование по ширинам рулонов и учитывает кромку как технологический отход.
    \item Порядок марок бумаги в выпуске полуфабриката задаётся технологически и рассматривается как входной параметр планирования.
    \item Параметры оборудования (скорости производства и резки, время перестановки ножей, время смены марки/плотности) считаются известными и постоянными на горизонте планирования; возможные колебания учитываются на этапе экспериментальной валидации.
    \item Склад моделируется укрупнённо ограничениями на вместимость и допустимую ширину; детальные аспекты адресного хранения и внутренней логистики не рассматриваются.
    \item Модель нацелена на построение выполнимого плана для набора заказов; вопросы долгосрочного стратегического планирования и прогнозирования спроса находятся вне рамок рассмотрения.
\end{itemize}

\subsection{Выводы по разделу}\label{subsec:theor_ch/problem_conclusion}
В разделе выделены элементы производственной цепочки (выпуск полуфабриката, резка, складирование) и сформулированы ключевые критерии качества плана. Далее в обзоре литературы рассматриваются подходы, позволяющие учитывать данные особенности в постановках задач раскроя и календарного планирования, а затем вводится формальная модель, которая используется в последующих главах для построения алгоритма и программной реализации.


\section{Обзор литературы}\label{sec:theor_ch/literature_review}

Цель данного обзора — показать развитие классической задачи линейного раскроя и связанных с ней постановок, а также выделить направления, наиболее близкие к задачам целлюлозно-бумажного производства: учёт потерь при продольной резке, ограничений по переналадке (перестановка ножей, смена схем), интеграция раскроя с календарным планированием и со складскими ограничениями. Особое внимание уделяется постановкам, в которых помимо минимизации отходов рассматриваются эксплуатационные критерии (число раскроев/операций и переналадок), поскольку именно они определяют практическую реализуемость производственного плана.

\subsection{Классическая задача линейного раскроя и её обобщения}\label{subsec:theor_ch/csp_classic}

Задача линейного раскроя относится к числу классических задач комбинаторной оптимизации и широко применяется при планировании производства в отраслях, где исходный материал поступает в виде заготовок фиксированной длины или ширины, а выходная продукция представлена наборами требуемых размеров. К таким отраслям относятся, в частности, металлургия, деревообработка, текстильная промышленность и целлюлозно-бумажное производство. Фундаментальные формальные постановки задачи линейного раскроя предложены П.~Гилмором и Р.~Гомори \cite{gilmore1961linear, gilmore1963linear}, где задача представлена в виде модели целочисленного линейного программирования, а также разработан подход генерации столбцов, позволяющий эффективно решать задачи большой размерности за счёт построения только необходимых схем раскроя.

Дальнейшее развитие исследований привело к формированию широкого класса задач раскроя и упаковки. Для систематизации постановок и выбора методов решения существенную роль сыграла типология, предложенная Х.~Дикхоффом \cite{dyckhoff1990typology}, а также её развитие и уточнение в более поздних работах, в частности в \cite{wascher2007typology}. Указанные классификации позволяют корректно отнести прикладные варианты раскроя, включающие дополнительные технологические ограничения, к соответствующим подклассам и тем самым обосновать применимость тех или иных методов решения.

С точки зрения вычислительной сложности задача раскроя относится к трудным комбинаторным задачам; на практике это означает необходимость компромисса между точностью и временем расчёта, особенно при росте числа заказов и расширении набора ограничений \cite{garey1979computers}. Поэтому в реальных производственных постановках широкое распространение получили либо точные методы с использованием декомпозиции, либо эвристические и метаэвристические алгоритмы, обеспечивающие приемлемое качество решений при ограниченном времени.

\subsection{Точные методы решения: декомпозиция и метод ветвей и цен}\label{subsec:theor_ch/exact_methods}

Для задачи линейного раскроя и её обобщений важную роль играют точные методы, основанные на декомпозиции и последовательном уточнении решения. Подходы, использующие декомпозицию Дантцига--Вульфа и генерацию столбцов, позволяют получать сильные нижние оценки и формировать решения в пространствах большой размерности. Практические аспекты построения алгоритмов типа «ветвей и цен» (ветвление в сочетании с генерацией столбцов), а также рекомендации по их эффективной реализации подробно обсуждаются в работе Ф.~Вандербека \cite{vanderbeck2000integer}. Современный обзор математических моделей и точных алгоритмов для задач раскроя и родственных задач упаковки представлен в \cite{delorme2016bin}, где рассматриваются типовые целочисленные формулировки, схемы построения раскроев и методы улучшения вычислительной устойчивости.

Точные методы особенно полезны для получения эталонных решений на задачах малой и средней размерности, а также для построения нижних оценок, позволяющих сравнивать качество эвристических решений. Однако при добавлении технологических ограничений (переналадки, очередность выполнения схем, складирование, сроки) размерность модели и сложность подзадач генерации колонок существенно возрастают, что ограничивает применение точных алгоритмов в режиме оперативного планирования.

\subsection{Приближённые методы: эвристики и метаэвристики}\label{subsec:theor_ch/heuristics}

В прикладных постановках, возникающих в промышленности, размерность задачи и число технологических ограничений зачастую делают применение точных методов ограниченным по вычислительным ресурсам. В связи с этим широко используются приближённые методы, в том числе эвристики (конструктивные и улучшающие), а также метаэвристики, включая локальный поиск и гибридные алгоритмические схемы. В работе Т.~Урбана \cite{urban1999solving} исследованы приближённые методы решения задачи раскроя на основе адаптивного поиска и показана эффективность подобного подхода для получения качественных решений при ограниченном времени расчёта. Существенная часть современных прикладных решений основывается на комбинации конструктивной генерации схем раскроя и процедур улучшения решения.

В качестве универсального семейства метаэвристик часто используются генетические алгоритмы, сформировавшиеся как общий подход к оптимизации в сложных пространствах решений \cite{goldberg1989genetic}. Для задач раскроя генетические алгоритмы привлекательны тем, что позволяют естественно кодировать схемы раскроя и порядок их выполнения, а также легко дополняются механизмами учёта отраслевых ограничений (переналадки, складирование, многокритериальность). Практическая эффективность генетического алгоритма для CSP демонстрировалась, в частности, в \cite{haessler1991}, а дальнейшие исследования развивали идеи многокритериальной оптимизации и настройки операторов эволюции под специфику производственных процессов \cite{bansal2003multiobjective,silva2008multiobjective,toksari2014multiobjective}.

\subsection{Потери материала при продольной резке и отраслевые постановки}\label{subsec:theor_ch/trim_loss}

Для целлюлозно-бумажной промышленности характерны постановки, ориентированные на минимизацию потерь материала при продольной резке рулонных заготовок, включая потери на обрезь и недоиспользование ширины. Обзор соответствующих постановок и методов решения, а также обсуждение практических аспектов минимизации потерь приведены в работе А.~Хинксмана \cite{hinxman1980trimloss}. В отличие от абстрактной постановки задачи линейного раскроя, отраслевые задачи минимизации потерь включают дополнительные компоненты: ограничения по настройке ножей, зависимость затрат от смены формата, наличие ограничений по складам и допустимой перепроизводству/недопроизводству, а также требования по срокам выполнения заказов.

Важным аспектом отраслевых задач является также порядок применения схем раскроя, поскольку изменение схемы ножей и форматные переходы приводят к простоям и дополнительным затратам. Поэтому на практике целесообразно учитывать не только состав схем раскроя, но и структуру последовательности их выполнения.

Отдельное направление составляет учёт затрат на переналадку (setup costs) и зависимость целевой функции от числа смен схем раскроя и перестановок ножей. Подобные постановки рассматриваются в работах, где расходы на переналадку включаются в оптимизационную модель наряду с отходом материала \cite{mobasher2013,martina2022}. Для производственных линий ЦБК такие компоненты имеют принципиальное значение, поскольку переналадка влияет не только на трудоёмкость, но и на пропускную способность участка резки.

\subsection{Интеграция раскроя с календарным планированием и очередностью выполнения работ}\label{subsec:theor_ch/integration_scheduling}

В производственных системах с несколькими взаимосвязанными стадиями (выпуск полуфабриката, резка, складирование и отгрузка) задача раскроя тесно связана с задачей формирования производственного расписания. Учет временных ограничений и переналадок оборудования приводит к необходимости совместного рассмотрения минимизации потерь материала и минимизации затрат времени/переналадок. Пример постановок календарного планирования операций резки на нескольких параллельных машинах представлен в \cite{giannelos2001parallel}. 

Подходы, ориентированные на одновременное решение задач минимизации потерь и формирования расписания в целлюлозно-бумажной отрасли, изучались, в частности, в \cite{harjunkoski1996trimloss, westerlund1998trimloss_scheduling}, где представлены целочисленные формулировки и обсуждены вычислительные особенности их применения. Отдельно рассматриваются постановки, в которых задача раскроя дополняется ограничениями на очередность выполнения схем раскроя; пример интегрированной постановки «раскрой + очередность» представлен в \cite{yanasse2007integrated}.

Для практического планирования существенную роль играют ограничения, которые связывают расписание резки с доступностью полуфабрикатов и состоянием склада, а также обеспечивают выполнимость планов при параллельной работе нескольких машин. В результате в рамках интегрированных постановок формируется не только набор схем раскроя, но и порядок их исполнения, согласованный с выпуском и логистикой полуфабрикатов, что особенно важно для непрерывных производств.

\subsection{Интеграция раскроя с планированием выпуска партий и многопериодные постановки}\label{subsec:theor_ch/integration_lotsizing}

При наличии связи между выпуском полуфабриката и последующей резкой, а также при ограничениях складирования возникает необходимость интеграции задачи раскроя с задачами планирования выпуска партий и управления запасами. Такой класс постановок рассматривается в работах, посвящённых совместному планированию объёмов производства и раскроя, например в \cite{nonas2000combined}. Современная классификация интегрированных постановок «планирование партий + раскрой», а также обзор литературы приведены в \cite{melega2018review}. Для целлюлозно-бумажной промышленности известны прикладные модели, в которых целевая функция учитывает одновременно затраты производства/хранения и потери при раскрое \cite{poltroniere2016tema}.

Интеграция с управлением запасами позволяет формализовать компромисс между немедленным выполнением резки и накоплением полуфабриката для повышения качества раскроя (снижение отхода и числа переналадок). В таких моделях возникают дополнительные ограничения по ёмкости складов, динамике запасов и синхронизации стадий производства, что напрямую соответствует реальным процессам на целлюлозно-бумажных предприятиях.

% --- начало расширенного раздела 1.1 ---
\paragraph{Колонковая генерация и линейное программирование.}
Классическим методом решения задачи линейного раскроя является линейно‑программная модель с генерацией колонок, предложенная Гилмором и Гомори в 1960‑х гг. \cite{gilmore1961linear,gilmore1963linear}. Алгоритм строит мастер‑проблему, переменные которой соответствуют допустимым схемам раскроя, и последовательно генерирует новые колонки (раскрои) с помощью задачи на рюкзак (динамического программирования). В дальнейшем подход был расширён: Вандербек предложил неявную генерацию колонок и комбинирование её с процедурой ветвления \cite{vanderbeck2000integer}; Делорме и соавт.\ провели сравнительный анализ эффективности методов \emph{branch-and-price} для задач раскроя и упаковки \cite{delorme2016bin}, а Янассе и Пинто изучили интеграцию раскроя с планированием работы оборудования \cite{yanasse2007integrated}. Эти работы показали, что колонковые методы хорошо работают на задачах средней размерности и пригодны для построения практически оптимальных решений.

\paragraph{Многокритериальные модели.}
Во многих практических ситуациях необходимо минимизировать одновременно несколько критериев (например, отходы бумаги, количество переналадок на станках, число схем раскроя). Бансал применил эволюционный алгоритм для одновременной минимизации отходов и числа раскроев, продемонстрировав на примерах, что многокритериальные алгоритмы позволяют находить компромиссные решения \cite{bansal2003multiobjective}. Силва и соавт.\ предложили генетический алгоритм для одноразмерной задачи CSP с двумя критериями – потерей материала и числом операций \cite{silva2008multiobjective}. Токсари и Байкасоглу разработали алгоритм имитации отжига, который учитывает и отходы, и время работы оборудования \cite{toksari2014multiobjective}. Такие исследования подчёркивают, что многокритериальные постановки более полно отражают реальные потребности производства ЦБК, где важно не только сократить отход, но и снизить количество переналадок и время выполнения заказов.

\paragraph{Эволюционные и эвристические подходы.}
Помимо линейно‑программных методов, значительное внимание уделяется эвристикам и метаэвристикам. Ещё в 1990‑х гг. Хесслер и Берда показали эффективность генетического алгоритма для задачи CSP \cite{haessler1991}. В последующих работах генетические алгоритмы успешно применялись к более сложным постановкам, включая задачи с изменяемым набором оборудования и расписанием \cite{genetics_1,genetics_2,evolution_1,evolution_2}. Обзор Мелега и соавт.\ подчёркивает, что эволюционные методы особенно полезны при большом числе технологических ограничений, когда методы линейного программирования становятся непрактичными \cite{melega2018review}. Таким образом, выбор метода зависит от размерности и сложности задачи: для небольших примеров эффективнее точные линейные методы, а при масштабных ограничениях – генетические алгоритмы, что будет подробно рассмотрено в последующих главах.
% --- конец расширенного раздела 1.1 ---


\subsection{Выводы по обзору и место настоящей работы}\label{subsec:theor_ch/lit_conclusion}

Таким образом, в литературе подробно изучены: (i) классическая задача линейного раскроя и методы её решения, (ii) отраслевые постановки минимизации потерь при продольной резке рулонных материалов, (iii) постановки, учитывающие очередность выполнения схем раскроя и переналадки, (iv) интеграция раскроя с задачами календарного планирования и планирования выпуска партий.

Вместе с тем для задач целлюлозно-бумажного производства сохраняется разрыв между моделями, ориентированными преимущественно на минимизацию потерь материала, и моделями, учитывающими согласованное формирование производственного расписания для взаимосвязанных стадий (выпуск полуфабриката, резка, складирование). Настоящая работа направлена на сокращение указанного разрыва за счёт разработки модели и алгоритмического подхода, обеспечивающих согласование решений по раскрою и формированию расписания с учётом технологических и временных ограничений.

Отдельно отметим, что в отечественных работах автора и соавторов рассматривались генетические и комбинированные подходы к одноразмерным задачам раскроя с учётом практических критериев качества. В частности, предложены алгоритмы генерации схем раскроя и подходы к настройке эвристик \cite{klimenko2024moscow}, исследованы генетические алгоритмы для постановок с допусками на объёмы выпуска \cite{klimenko2024_prin}, а также рассмотрены методы, ориентированные на одновременное снижение отходов и числа перестановок ножей \cite{klimenko2025ivdon,klimenko2025fruct}. Представленная в диссертации постановка развивает данные направления за счёт явного согласования стадии выпуска полуфабриката, операций резки и ограничений складирования в рамках единого плана.

\FloatBarrier
\section{Постановка задачи}
В данном разделе вводятся обозначения и структура решения, после чего фиксируются критерии оптимальности и ограничения выполнимости. Изложение организовано так, чтобы последующие разделы алгоритмической и программной частей могли ссылаться на единый набор терминов и параметров.
\subsection{Обозначения}
Обозначим условия задачи:

\subsubsection{Индексы и диапазоны}\label{subsubsec:theor_ch/index_ranges}
Ниже используются стандартные индексы: индексы заказов (элементы множества заказов), индексы машин и складов (элементы соответствующих множеств оборудования), а также индексы позиции внутри плана и индексы форматов в раскрое. В каждом случае диапазон индекса определяется тем множеством или структурой данных, к которому он относится.

\subsubsection{Таблицы обозначений}\label{subsubsec:theor_ch/notation_tables}
Для удобства восприятия введённые обозначения сгруппированы в таблицах. Таблицы не заменяют последующие формальные определения, а служат кратким справочником.

\begin{table}[htbp]
\centering
\caption{Основные множества и параметры (входные данные модели)}%
\label{tab:theory_notation_input}%
\begin{tabular}{|p{0.22\linewidth}|p{0.70\linewidth}|}
\hline
\textbf{Обозначение} & \textbf{Смысл} \\
\hline
$O$ & множество заказов (ширины, объёмы, сроки, марка, плотность, слои и др.) \\
\hline
$M$ & множество бумагоделательных машин (БДМ) \\
\hline
$C$ & множество продольно-резательных и бобино-резательных станков (ПРС/БРС) \\
\hline
$T$ & множество складов полуфабрикатов \\
\hline
$B$ & времена установки/переходов марок бумаги (порядок марок задаётся) \\
\hline
$H$ & время начала запуска производства \\
\hline
$V$ & размер кромки тамбура \\
\hline
\end{tabular}
\end{table}

\begin{table}[htbp]
\centering
\caption{Основные структуры решения (выход модели)}%
\label{tab:theory_notation_output}%
\begin{tabular}{|p{0.22\linewidth}|p{0.70\linewidth}|}
\hline
\textbf{Обозначение} & \textbf{Смысл} \\
\hline
$\overline{\overline{X}}$ & общий план производства бумаги на БДМ \\
\hline
$\overline{\overline{Y}}$ & общий план нарезки (раскроев) на ПРС/БРС \\
\hline
$\overline{\overline{Z}}$ & очередь событий, определяющая порядок и времена выполнения выпусков и нарезок \\
\hline
$G$ & глобальный план, объединяющий выпуск, нарезку и очередь событий \\
\hline
\end{tabular}
\end{table}

$B = \{ b_{1},\ \ldots,\ b_{U}\}$ -- множество времен  установок марок бумаги, где $f$ -- номер марки бумаги и $b_{f}$ -- время установки первой марки при $f = 1$ или время перехода с марки $f - 1$ на марку $f$. $U$ -- количество марок. Порядок марок соответствует заданному.

$O = \left\{ o_{1},\ \ldots,\ o_{Q} \right\}$ -- множество заказов, где $o_{i} = \left( w_{i},\ q_{i},\ h_{i},\ f_{i},\ p_{i},\ l_{i},\ g_{i} \right)$ -- заказ, имеющий ширину $w_{i}$, количество $q_{i}$, срок $h_{i}$ - дата и время, номер марки бумаги $f_{i} \in \lbrack 1,U\rbrack$, плотность $p_{i}$ и количество слоев $l_{i}$ и индикатор является ли заказ, выпускаемым БРС $g_{i} = \{ 0,1\}$. $Q$ -- количество заказов.

$H$ -- дата и время начала запуска производства.

$V$ -- размер кромки тамбура.

Условия, касающиеся оборудования:

$M = \{ m_{1},\ \ldots,\ m_{I}\}$ -- множество бумагоделательных машин (БДМ), 
где $m_{i}\  = ({w^m}_{i},\ {v^m}_{i},\ {g^m}_{i})$ -- БДМ, имеющая ширину ${w^m}_{i}$, 
время производства одного рулона ${v^m}_{i}$ и время смены плотности на 1 грамм ${g^m}_{i}$. $I$ -- количество БДМ.

$C = \{ c_{1},\ \ldots,\ c_{J}\}$ -- множество продольно-резательных станков (ПРС), где $c_{i} = ({w^c}_{i},\ {n^c}_{i},\ {v^c}_{i},\ {k^c}_{i},\ {g^c}_{i})$ -- ПРС, имеющий максимальную ширину тамбура ${w^c}_{i}$, количество ножей ${n^c}_{i}$, время нарезки одного слоя для одного рулона ${v^c}_{i}$, время перестановки одного ножа ${k^c}_{i}$ и индикатор является ли БРС ${g^c}_{i} = \{ 0,1\}$. $J$ -- количество ПРС.

$T\  = \ \{ t_{1},\ldots,\ t_{K}\}$ -- множество складов, где $t_{i} = ({r^T}_{i},{u^T}_{i})$ -- склад, имеющий ширину ${r^T}_{i}$ и максимальное количество рулонов ${u^T}_{i}$. $K$ -- количество складов.

Обозначим переменные:

$\overline{\overline{X}} = \{{\overline{X}}_{1},\ldots,{\overline{X}}_{I}\}$ -- общий план производства бумаги. ${\overline{X}}_{i \in \lbrack 1,\ I\rbrack}$ -- план производства для $i$-ой БДМ, 
состоящий из тамбуров $X_{i,j} \in {\overline{X}}_{i \in \lbrack 1,\ I\rbrack}$, 
которые необходимо произвести. $X_{ij} = ({{w^X}}_{i,j},\ {{f^X}}_{i,j},\ {{p^X}}_{i,j})$ -- тамбур, имеющий ширину совпадающую с шириной БДМ ${{w^X}}_{i,j} = {w^m}_{i}$, номер марки бумаги ${f^X} \in \lbrack 1,U\rbrack$ и плотность ${p^X}$. 

$\overline{\overline{Y}} = \{{\overline{Y}}_{1},\ldots,{\overline{Y}}_{J}\}$ -- общий план нарезки бумаги. ${\overline{Y}}_{i \in \lbrack 1,J\rbrack}$ -- план нарезки для $i$-ой ПРС, состоящий из раскроев $Y_{i,j} \in {\overline{Y}}_{i \in \lbrack 1,\ J\rbrack}$, которые необходимо выпустить. $Y_{i,j} = ({{y^Y}}_{i,j},\ {{f^Y}}_{i,j},\ {{p^Y}}_{i,j},\ {{l^Y}}_{i,j},\ {{g^Y}}_{i,j})$ -- раскрой, имеющий упорядоченный набор форматов $y_{i,j,k} \in {{y^Y}}_{i,j}$ (где формат $y_{i,j,k} = ({r^Y}_{i,j,k},{{g^Y}}_{i,j,k})$ состоит из ширины ${r^Y}_{i,j,k}$ и номера заказа ${{g^Y}}_{i,j,k} = \left\{ 0 \right\} \cup \lbrack 1,U\rbrack$, если номер заказа ${{g^Y}}_{i,j,k} = 0$, то нарезаемый формат -- тамбур для БРС, а не является конечным заказом), номер марки бумаги ${{f^Y}}_{i,j} \in \lbrack 1,U\rbrack$, плотность ${{p^Y}}_{i,j}$ и количество слоев ${{l^Y}}_{i,j}$. 

Количество перестановок ножей для $i$-ой ПРС и $j$-го плана различается при первом раскрое $j = 1$ и последующих раскроев $j > 1$. При первом раскрое $j = 1$ необходимо выставить ножи ПРС перед первым форматом и после каждого формата $\sigma(i,j) = \left| y_{i,j} \right| + 1$. Для последующих раскроев $j > 1$ количество ножей соответствует сумме предыдущих перестановок, разности количества форматов и количества совпадающих форматов в началах текущего раскроя и предыдущего раскроя $\sigma(i,j) = \sigma(i,j - 1) + \left| y_{i,j} \right| - \max_{n}\left\{ 0 \leq n \leq min\left\{ \left| y_{i,j - 1} \right|,\left| y_{i,j} \right| \right\}:\ \forall k \leq n\ \ {r^Y}_{i,j - 1,k} = {r^Y}_{i,j,k} \right\}$.

$\overline{\overline{Z}}\ $-- очередь событий, состоящая из событий $\overline{Z} \in \overline{\overline{Z}}$. $\overline{Z} = (h,\ \overline{h},\ e,d,z,\overline{T},\overline{\overline{T}},\ \overline{t'})$, где:

\begin{itemize}
    \item $e\  = \ \{ 0,\ 1\}$ -- тип события, при $e = 0$ -- это событие по выполнению плана выпуска тамбура, при $e = 1$ -- это событие по нарезки на ПРС.
    \item $d$ -- номер машины для выполнения плана
    $$d \in \left\{ 
    \begin{array}{r}
    I,e = 0 \\
    J,\ e = 1
    \end{array} 
    \right.\ $$
    \item $z$ -- номер плана по выпуску тамбура $(e = 0)$ или нарезке $(e = 1)$ для $d$-ой машины.
    
    \item $h$ -- время начала события.
    \item $\overline{h}$ -- время конца события. При выпуске тамбура $(e=0)$ время конца $\overline{h}$ события зависит от начала события $h$, времени работы на БДМ ${v^m}_{d}$, плотности бумаги ${g^m}_{d}$ и скорости смены марки $b_f$ (если требуется замена), где $f={p^X}_{d,z}$ - номер марки. Время окончания выпуска тамбура составляет:
    $$
    {
        \overline{h}=\left\{        
        \begin{array}{r}
                 h+{v^m}_{d}+{g^m}_{d}({p^X}_{d,z}-{p^X}_{d,z-1})+b_fI(f={f^X}_{d,z-1}),\\z>0,e=0\\
                 h+{v^m}_{d}+{g^m}_{d}{p^X}_{d,z}+b_f,z=0,e=0
        \end{array}
        \right.
    }
    $$. При нарезке $(e=1)$ время зависит от скорости работы ПРС ${v^c}_{d}$, количества слоев ${{l^Y}}_{d,z}$ и скорости ${k^c}_{d}$, количества $\sigma(d,z)$ перестановок ножей. Время окончания нарезки рулона составляет:
    $$
    {
        \overline{h}=\left\{        
        \begin{array}{r}
                 h+{v^c}_{z}{{l^Y}}_{d,z}+{k^c}_{d}(\sigma(d,z)-\sigma(d,z-1)),z>0,e=1\\
                 h+{v^c}_{z}{{l^Y}}_{d,z}+{k^c}_{d}\sigma(d,z),z=0,e=1
        \end{array}
        \right.
    }
    $$.
    
    \item $\overline{T} = \{{\overline{t}}_{1},\ldots,{\overline{t}}_{K}\}$ -- содержимое склада тамбуров. $K$ -- количество складов. Содержимое $i$-го склада состоит из набора тамбуров, полученных при выполнении планов по производству бумаги ${\overline{t}}_{i} = \{{\overline{t}}_{i,j}\}$. ${\overline{t}}_{i,j} = ({\overline{w^T}}_{i,j},\ {\overline{f^T}}_{i,j},\ {\overline{p^T}}_{i,j})$ -- тамбур на складе, имеющем ширину ${\overline{w^T}}_{i,j}$, марку ${\overline{f^T}}_{i,j}$ и плотность ${\overline{p^T}}_{i,j}$. При событии по производству бумаги $(e = 0)$ рулон из плана не будет сразу включен в содержимое склада для текущего события, так как будет произведен в будущем. Когда наступит событие со временем начала $h$ не меньше времени окончания текущего события $\overline{h}$, то данный рулон будет включен в склад $\overline{T}$. При нарезке $(e = 1)$, тамбуры отправляемые на нарезку не включены. Текущее содержимое складов переходит к следующему событию, но без рулонов, отправленных на нарезку, и с новопроизведенными рулонами.
    \item $\overline{\overline{T}} = \{{\overline{\overline{t}}}_{1},\ldots,{\overline{\overline{t}}}_{K}\}$ -- запланированное содержимое склада тамбуров ${\overline{\overline{t}}}_{i} = \{{\overline{\overline{t}}}_{i,j}\}$, включая текущее содержимое $\overline{T} \subset \overline{\overline{T}}$ и тамбуры, которые в момент события находятся на производстве БДМ. $K$ -- количество складов. ${\overline{\overline{t}}}_{i,j} = ({\overline{\overline{w^T}}}_{i,j},\ {\overline{\overline{f^T}}}_{i,j},\ {\overline{\overline{p^T}}}_{i,j})$ -- тамбур на складе, имеющем ширину ${\overline{\overline{w^T}}}_{i,j}$, марку ${\overline{\overline{f^T}}}_{i,j}$ и плотность ${\overline{\overline{p^T}}}_{i,j}$. При событии по производству бумаги $(e = 0)$ рулон из плана будет включен в содержимое склада для текущего события. При нарезке $(e = 1)$, тамбуры отправляемые на нарезку не включены. Текущее планируемое содержимое складов переходит к следующему событию, но без рулонов, отправленных на нарезку.
    
    \item $\overline{t'} = \{{\overline{t''}}_{i}\}$ -- использованные тамбуры при нарезке $e = 1$, при производстве бумаги отсутствуют используемые тамбуры $e = 0:\overline{t'} = \varnothing$. ${\overline{t''}}_{i} = ({\overline{w^t}}_{i},\ {\overline{f^t}}_{i},\ {\overline{p^t}}_{i})$ -- используемый тамбур, имеющий ширину ${\overline{w^t}}_{i}$, марку ${\overline{f^t}}_{i}$ и плотность ${\overline{p^t}}_{i}$.
\end{itemize}



$G = (\overline{\overline{X}},\overline{\overline{Y}},\overline{\overline{Z}})$ -- глобальный план, состоящий из плана производства бумаги, плана нарезки бумаги и очереди событий.

\subsection{Целевая функция и критерии оптимизации}\label{subsec:theor_ch/objective}
Оптимальность плана можно измерить по следующим критериям. Порядок перечисления соответствует приоритету критериев:
\begin{enumerate}
    \item суммарная ширина отходов;
    \item максимальное количество раскроев из планов для каждой ПРС;
    \item общее количество перестановок ножей.
\end{enumerate}

Далее вводится векторная целевая функция $\mu(G)$ и даётся интерпретация её компонент, соответствующих указанным критериям.


% --- начало расширенного описания целевой функции ---
Векторная целевая функция
\[
\mu(G) = \Bigl\langle \omega(G),\, \varphi(G),\, \tau\bigl(\overline{\overline{Y}}\bigr) \Bigr\rangle \rightarrow \min
\]
отражает три ключевых аспекта качества глобального плана $G=(\overline{\overline{X}},\overline{\overline{Y}},\overline{\overline{Z}})$.

\paragraph{Минимизация отходов ($\omega(G)$).}
Суммарная ширина отходов определяется как разность между суммарной шириной использованных тамбуров и суммарной шириной раскроев:
\[
\omega(G)\;=\;\sum_{\overline{Z}\in\overline{\overline{Z}}:\,e=1}
\Biggl(\,\sum_{i=1}^{l^Y_{d,z}}\overline{w^t}_{i}\;-\;\sum_{i=1}^{|{y^Y}_{d,z}|} r^Y_{d,z,i}\Biggr),
\]
где $\overline{w^t}_{i}$ — ширины тамбуров, а $r^Y_{d,z,i}$ — ширины отдельных форматов в раскроях. Этот критерий оценивает суммарный отход материала и включает в себя кромку; его минимизация обеспечивает экономию сырья и снижение экологических издержек \cite{haessler1991,mobasher2013}. В многокритериальных работах по задаче раскроя минимизация потерь часто рассматривается как основной показатель качества решения \cite{bansal2003multiobjective}.

\paragraph{Сбалансированная загрузка ПРС ($\varphi(G)$).}
Вторая компонента
\[
\varphi(G)\;=\;\max_{\overline{Z}\in\overline{\overline{Z}}:\,e=1} z
\]
определяет максимальное число раскроев на одной ПРС. Число раскроев для каждой машины прямо пропорционально трудоёмкости и времени выполнения заказов. Ограничение пиковых нагрузок позволяет равномернее распределить работу между машинами, снизить время ожидания и обеспечить более стабильную эксплуатацию оборудования \cite{cerqueira2021,vanderbeck2000}.

\paragraph{Минимизация перестановок ножей ($\tau(\overline{\overline{Y}})$).}
Третья компонента
\[
\tau\bigl(\overline{\overline{Y}}\bigr)\;=\;\sum_{i=1}^{J} \sigma\bigl(i,\,|\overline{Y}_i|\bigr),
\]
где $\sigma(i,j)$ — количество перестановок ножей при $j$‑м раскрое на $i$‑й ПРС, отражает общую трудоёмкость переналадок. Перестановка ножей требует времени и сопровождается простоем, поэтому минимизация $\tau(\overline{\overline{Y}})$ помогает уменьшить длительность производственного цикла \cite{cyberleninka_slitting_params,mobasher2013}.

\paragraph{Интерпретация критериев в единицах и KPI предприятия.}
Компоненты $\mu(G)$ могут быть интерпретированы в терминах производственных показателей. Величина $\omega(G)$ выражается в единицах ширины (в используемых единицах измерения входных данных) и на практике может быть дополнительно приведена к относительному показателю, например к доле отхода в процентах от суммарной ширины использованного полуфабриката. Показатель $\varphi(G)$ измеряется количеством операций нарезки (раскроев) и отражает загрузку участка резки; при прочих равных уменьшение $\varphi(G)$ снижает риск возникновения «пиков» по времени выполнения и уменьшает число технологических переходов. Показатель $\tau(\overline{\overline{Y}})$ измеряется количеством перестановок ножей и напрямую связан с простоем оборудования и трудоёмкостью переналадки.

В прикладной интерпретации данные показатели часто рассматриваются совместно: план с минимальным отходом может приводить к большому числу раскроев и переналадок (что ухудшает производительность), тогда как план с упрощённой последовательностью схем уменьшает переналадки, но может увеличивать отход. Поэтому сравнение альтернативных планов удобно выполнять как по формальным критериям, так и по производственным KPI (например, доля отхода, количество операций на смену, суммарное время переналадок).

\paragraph{Учёт многокритериальности.}
Существует несколько подходов к оптимизации векторной функции $\mu(G)$:
\begin{itemize}
  \item \emph{Лексикографическая минимизация} — критерии упорядочиваются по важности, и решение выбирается по принципу последовательного улучшения: сначала минимизируется $\omega(G)$, затем $\varphi(G)$, и только в случае равенства первых двух критериев минимизируется $\tau(\overline{\overline{Y}})$. Такой приоритетный подход оправдан в задачах ЦБК, где уменьшение отходов имеет наибольшее экономическое значение, а балансировка загрузки и сокращение перестановок позволяют повысить операционную эффективность \cite{haessler1991,cerqueira2021,vanderbeck2000}.
  \item \emph{Взвешенная сумма} — комбинирование критериев в скалярную функцию $\omega(G) + \alpha\,\varphi(G) + \beta\,\tau(\overline{\overline{Y}})$ с весовыми коэффициентами $\alpha,\beta>0$. Подобный подход применяется в многокритериальных генетических алгоритмах для CSP, где веса позволяют пользователю задавать компромисс между расходом материала и производственными издержками \cite{bansal2003multiobjective,silva2008multiobjective}. Подбор весов существенно влияет на найденные решения и обычно определяется экспертом или на основе экспериментов.
  \item \emph{Методы попарного сравнения и метаэвристики} — например, имитация отжига с несколькими целями \cite{toksari2014multiobjective} или эволюционные алгоритмы, где каждая особь оценивается по нескольким критериям, а отбор осуществляется на основе доминирования (Pareto‑optimality). Эти методы позволяют находить набор несравнимых решений, из которых можно выбрать подходящее по предпочтениям заказчика.
\end{itemize}

Выбор конкретного способа агрегирования критериев зависит от требований предприятия и особенностей производства. В настоящей работе используется лексикографическое упорядочение критериев, что обеспечивает жёсткий приоритет минимизации отходов и упрощает сравнение альтернативных планов. 
% --- конец расширенного описания целевой функции ---

\subsection{Ограничения}
\begin{enumerate}
    \item Порядок марок в производстве тамбуров должен соответствовать заданному $\forall i,j:\ {{f^X}}_{i,j} \leq \ {{f^X}}_{i,j + 1}$.

    Данное ограничение фиксирует технологически заданную очередность марок бумаги и исключает решения, в которых последовательность марок нарушается.

    \item Для раскроя его сумма ширин заказов и кромки тамбура не должна превышать ширину ПРС $V + \sum_{k = 1}^{\left| y_{i,j} \right|}{r^Y}_{i,j,k} \leq {w^c}_{i},i=\overline{1,Q},j=\overline{1,U}$. Общий план нарезки бумаги должен удовлетворять всем заказам $\forall n \in \lbrack 1,Q\rbrack\sum_{}^{}{I\left( {{g^Y}}_{i,j,k} = n \right) = q_{n}}$.

    Первая часть ограничивает допустимость каждого раскроя по ширине (с учётом кромки), а вторая гарантирует выполнение объёмов всех заказов в рамках общего плана нарезки.

    \item Очередь событий определяет последовательность выполнения планов производства бумаги и нарезки, поэтому количество событий равно сумме всех планов для каждой БДМ и ПРС $\left| \overline{\overline{Z}} \right| = \sum_{i = 1}^{I}\left| {\overline{X}}_{i} \right| + \sum_{i = 1}^{J}\left| {\overline{Y}}_{i} \right|$.

    Это условие обеспечивает полноту расписания: каждой операции выпуска и каждой операции нарезки соответствует событие в очереди, и тем самым план допускает временную интерпретацию.

    \item Для каждого заказа срок должен быть не раньше чем время конца события, в котором есть нарезка соответствующего заказа $\forall\overline{Z},\forall i \in \lbrack 1,Q\rbrack:\overline{h} \leq h_{i}$.

    Ограничение связывает план с дедлайнами: выполнение заказов допускается только в том случае, если соответствующие операции нарезки завершаются не позже заданных сроков.

    \item Для любого тамбура из плана по производству бумаги должно быть соответствующее событие в очереди событий $e = 0,\forall X_{i \in \lbrack 1,\ I\rbrack,j \in \left\lbrack 1,\left| {\overline{X}}_{i} \right| \right\rbrack} \in \overline{X}\ \exists d = i,z = j$.

    Данное условие исключает ситуации, когда тамбур присутствует в плане выпуска, но не включён в расписание событий, то есть фактически не производится.

    \item Для любого раскроя из плана по нарезке бумаги должно быть соответствующее событие в очереди событий $e = 1,\forall Y_{i \in \lbrack 1,J\rbrack,j \in \left\lbrack 1,\left| {\overline{Y}}_{i} \right| \right\rbrack} \in \overline{Y}\ \exists d = i,z = j$.

    Аналогично предыдущему, это ограничение гарантирует, что каждая операция нарезки из плана действительно запланирована во времени.

    \item Для осуществления нарезки $e = 1$ должны присутствовать необходимые тамбуры на складе в количестве равному числу слоев $\forall\overline{Z}\sum_{i,j}^{}{I\left( \sum_{k}^{}{r^Y}_{d,z,k} + V \leq {\overline{w^T}}_{i,j}\&{{f^Y}}_{d,z} = \ {\overline{f^T}}_{i,j}\&{{p^Y}}_{d,z} = {\overline{p^T}}_{i,j} \right)} = \ {{l^Y}}_{d,z}$.

    Ограничение обеспечивает материальную выполнимость операции нарезки: к моменту события на складе должны находиться подходящие тамбуры, совместимые по ширине, марке и плотности, причём их число соответствует требуемому числу слоёв.

    \item В содержимом складов тамбуров $\overline{T}$ тамбуры не должны превышать ограничения соответствующего склада по ширине ${\overline{w^T}}_{i,j} \leq {r^T}_{i}\forall{\overline{t}}_{i,j} \in {\overline{t}}_{i}$ и по количеству $\left| {\overline{t}}_{i} \right| \leq {u^T}_{i}$.

    Данное условие ограничивает текущее состояние склада: на складе не может находиться полуфабрикат шире допустимого и сверх заданной ёмкости.

    \item Для запланированного содержимого складов $\overline{\overline{T}}$ тамбуры не должны превышать ограничения соответствующего склада по ширине ${\overline{\overline{w^T}}}_{i,j} \leq {r^T}_{i}\forall{\overline{\overline{t}}}_{i,j} \in {\overline{\overline{t}}}_{i}$ и по количеству $\left| {\overline{\overline{t}}}_{i} \right| \leq {u^T}_{i}$.
 
    Это ограничение распространяет требования вместимости на планируемое содержимое склада, включая полуфабрикаты, которые уже запущены в производство и будут поступать на склад в будущем.
\end{enumerate}
\subsection{Схема производственного потока и трассируемость ограничений}\label{subsec:theor_ch/traceability}
Для удобства чтения постановки полезно явно зафиксировать, как элементы глобального плана связаны с этапами производственной цепочки и какими данными обеспечивается выполнение каждого ограничения.

\begin{table}[htbp]
\centering
\caption{Укрупнённая схема потока полуфабриката и операций}%
\label{tab:theory_process_flow}%
\begin{tabular}{|p{0.22\linewidth}|p{0.70\linewidth}|}
\hline
\textbf{Этап} & \textbf{Роль в глобальном плане} \\
\hline
Выпуск полуфабриката (БДМ) & Формирование $\overline{\overline{X}}$ и соответствующих событий $e=0$ в $\overline{\overline{Z}}$ (порядок марок, длительность выпуска) \\
\hline
Складирование (склады полуфабриката) & Динамика $\overline{T}$ и $\overline{\overline{T}}$ как доступность тамбуров для последующей нарезки при ограничениях вместимости \\
\hline
Нарезка (ПРС/БРС) & Формирование $\overline{\overline{Y}}$ и событий $e=1$ в $\overline{\overline{Z}}$ (схемы раскроя, переналадки, выполнение заказов) \\
\hline
Выполнение заказов & Обеспечение требуемых объёмов по заказам и соблюдение сроков за счёт согласования выпуска, склада и операций резки \\
\hline
\end{tabular}
\end{table}

\begin{table}[htbp]
\centering
\caption{Трассируемость ограничений: смысл и требуемые данные}%
\label{tab:theory_constraint_traceability}%
\begin{tabular}{|p{0.05\linewidth}|p{0.36\linewidth}|p{0.49\linewidth}|}
\hline
\textbf{№} & \textbf{Практический смысл} & \textbf{Какие данные/параметры нужны} \\
\hline
1 & Технологическая очередность марок бумаги в выпуске & Порядок марок, значения ${{f^X}}_{i,j}$, технологические правила смены марки \\
\hline
2 & Допустимость раскроя по ширине и выполнение объёмов заказов & $V$, ширины форматов $r^Y_{\cdot}$, ширина станка ${w^c}_{i}$, объёмы $q_n$, привязка форматов к заказам ${{g^Y}}_{\cdot}$ \\
\hline
3 & Полнота расписания: каждому выпуску и раскрою соответствует событие & Размерности планов $|\overline{X}_i|$, $|\overline{Y}_i|$, количество БДМ $I$ и станков $J$ \\
\hline
4 & Соблюдение сроков исполнения заказов & Сроки $h_i$, времена событий $h$ и $\overline{h}$, правило сопоставления раскроев заказам \\
\hline
5 & Наличие событий для всех операций выпуска & Идентификаторы операций в $\overline{X}$ и механизм формирования событий $e=0$ \\
\hline
6 & Наличие событий для всех операций нарезки & Идентификаторы операций в $\overline{Y}$ и механизм формирования событий $e=1$ \\
\hline
7 & Материальная выполнимость нарезки (наличие подходящих тамбуров) & Текущее содержимое склада $\overline{T}$: ширина/марка/плотность, число слоёв ${{l^Y}}_{d,z}$, параметры раскроя и совместимости \\
\hline
8 & Непереполнение склада в текущем состоянии & Ограничения склада ${r^T}_{i}$ и ${u^T}_{i}$, описание $\overline{T}$ \\
\hline
9 & Непереполнение склада в планируемом состоянии (с учётом запуска) & Ограничения склада ${r^T}_{i}$ и ${u^T}_{i}$, описание $\overline{\overline{T}}$ \\
\hline
\end{tabular}
\end{table}
\subsection{Иллюстративный пример интерпретации плана}\label{subsec:theor_ch/example}
Для пояснения смысла введённых структур рассмотрим небольшой пример. Пусть имеется одна БДМ, один станок резки и один склад, а также несколько заказов, различающихся по ширине и объёму. Тогда план выпуска задаёт последовательность тамбуров, которые необходимо изготовить, план нарезки задаёт набор схем раскроя (каждая схема определяет, какие форматы одновременно получают из одного тамбура), а очередь событий фиксирует порядок выполнения выпусков и нарезок с учётом доступности полуфабрикатов.

В таком примере можно интерпретировать глобальный план как цепочку решений: сначала выбирается, какие тамбуры следует изготовить и в какой последовательности, затем определяется, какие тамбуры и когда направлять на резку, а также какие партии целесообразно накопить на складе для уменьшения отходов и числа переналадок. При этом ключевым является баланс между критериями качества: более «плотные» раскрои уменьшают отход, но могут увеличивать число смен схем и усложнять расписание.

\subsection{Проверка корректности модели и критерии выполнимости}\label{subsec:theor_ch/validation}
При практическом применении модели полезно проверять выполнимость полученного плана по набору инвариантов. Такие проверки позволяют выявлять ошибки данных и некорректные решения на раннем этапе.
\begin{itemize}
    \item Выполнение заказов: объёмы выпуска по каждому заказу должны соответствовать требованиям (каждый заказ удовлетворён в полном объёме).
    \item Выполнимость раскроев: для каждой операции нарезки соблюдены ограничения по ширине и по технологическим параметрам станка.
    \item Наличие полуфабриката: операции нарезки выполняются только при наличии требуемых тамбуров на складе (с учётом числа слоёв).
    \item Непереполнение складов: текущее и планируемое содержимое складов не нарушает ограничений ёмкости и допустимой ширины.
    \item Соблюдение сроков: все операции, относящиеся к заказам, завершаются не позже заданных сроков.
\end{itemize}
Указанные проверки не изменяют модель, но обеспечивают интерпретируемость и воспроизводимость результатов, что важно при реализации и экспериментальной валидации.

\subsection{Особые случаи и граничные условия}\label{subsec:theor_ch/boundary_cases}
В реальном планировании возможны ситуации, которые требуют отдельной интерпретации результатов модели и алгоритма.
\begin{itemize}
    \item Отсутствие части заказов или нулевые объёмы: модель должна корректно формировать план, не создавая лишних операций выпуска и нарезки.
    \item Дефицит складской вместимости: при малых значениях ${u^T}_{i}$ или жёстких ограничениях по ширине ${r^T}_{i}$ возрастает роль синхронизации выпуска и нарезки; в таких условиях особенно важно избегать накопления полуфабриката.
    \item Многослойные заказы: требование наличия нескольких совместимых тамбуров для одной операции нарезки усиливает зависимость плана резки от структуры выпуска и от состояния склада.
    \item Ограничения по ножам и переналадкам: при большом числе узких форматов может увеличиваться $\tau(\overline{\overline{Y}})$; на практике это проявляется как рост времени переналадок и снижение пропускной способности участка резки.
    \item Невыполнимость по срокам: если при текущих ресурсах и ограничениях выполнить все заказы в срок невозможно, то планирование следует дополнять процедурами диагностики (выявление «узких мест») и правилами приоритизации (какие заказы допускается переносить).
\end{itemize}
Включение рассмотрения граничных условий повышает устойчивость применения модели: позволяет заранее определить, какие конфигурации входных данных приводят к невыполнимости или требуют изменения технологических правил и параметров.

Для решения задачи необходимо найти план, который будет иметь наилучшее значение целевой функции и при этом соответствовать ограничениям $G:\mu(G) \rightarrow \min$.

\subsection{Источники данных и калибровка параметров}\label{subsec:theor_ch/data_calibration}
Практическая применимость модели определяется тем, насколько корректно заданы входные параметры, описывающие оборудование и технологические ограничения. В условиях промышленного предприятия эти параметры, как правило, извлекаются из нескольких источников: производственные регламенты и технологические карты (номинальные скорости и допуски), исторические данные MES/ERP (фактические времена выполнения операций, объёмы и сроки), а также данные АСУ ТП/SCADA (временные метки, состояния оборудования, простои). Для складов параметры вместимости и ограничения по ширине формируются по паспортным данным и фактическим правилам хранения.

При подготовке данных целесообразно выполнять калибровку параметров под конкретный участок:
\begin{itemize}
    \item сопоставлять номинальные и фактические скорости выпуска и резки, выделяя типовые режимы по маркам бумаги и плотности;
    \item оценивать характерные времена переналадок (перестановки ножей, форматные переходы) по статистике прошлых смен;
    \item фиксировать и регулярно актуализировать ограничения склада (ёмкость, допустимые ширины, правила размещения), поскольку фактическая доступная вместимость может зависеть от организации хранения;
    \item согласовывать правила приоритизации заказов и допустимые отступления от сроков с производственными службами, чтобы критерии оптимальности соответствовали целям планирования.
\end{itemize}
Отдельное внимание следует уделять качеству исходных данных: пропуски и ошибки в заказах, некорректные ширины или плотности, а также несогласованность справочников марок приводят к невыполнимости плана независимо от выбранного алгоритма. Поэтому на практике полезно выполнять автоматизированную проверку входных данных перед запуском оптимизации, а результаты калибровки фиксировать в виде набора настраиваемых параметров, обновляемых при изменении оборудования или регламентов.

\subsection{Размерность, вычислительная сложность и масштабируемость}\label{subsec:theor_ch/scalability}
Рассматриваемая постановка относится к классу трудных комбинаторных задач: уже классическая задача линейного раскроя является NP-трудной, а добавление очередности, складских ограничений и переналадок повышает сложность за счёт взаимосвязи решений по выпуску, резке и расписанию. Поэтому при росте размерности задачи ключевым становится выбор алгоритмического подхода, обеспечивающего приемлемое качество плана за ограниченное время.

С практической точки зрения размерность определяется, прежде всего, количеством заказов $Q$, числом марок бумаги $U$, количеством оборудования $I$ (БДМ) и $J$ (станки резки), а также длиной планов $|\overline{X}_i|$ и $|\overline{Y}_i|$. Увеличение $Q$ и разнообразия форматов приводит к росту числа потенциальных схем раскроя и вариантов их последовательности, а увеличение $I$ и $J$ расширяет пространство распределения операций между машинами. Дополнительный вклад в сложность вносят складские ограничения, поскольку при построении очереди событий требуется постоянно отслеживать доступность полуфабриката и недопускать переполнения складов.

Для обеспечения масштабируемости на практике применяются следующие подходы:
\begin{itemize}
    \item укрупнение данных (агрегация близких по параметрам заказов, группировка партий) на этапе построения плана с последующим уточнением;
    \item ограничение множества рассматриваемых схем раскроя за счёт правил генерации и фильтрации (например, отбрасывание заведомо неэффективных схем по отходу и по числу ножей);
    \item декомпозиция по времени или по маркам бумаги (построение плана по сегментам с согласованием на границах);
    \item использование эвристических и метаэвристических методов, которые позволяют учитывать ограничения в виде процедур проверки выполнимости и улучшения решения без полного перебора.
\end{itemize}
В последующих главах при разработке алгоритма акцент делается на том, чтобы проверки выполнимости и оценка критериев могли выполняться быстро, а также чтобы решение можно было получать в режиме оперативного планирования при характерных для предприятия размерах $Q$, $I$ и $J$.

\subsection{Выводы по постановке задачи}\label{subsec:theor_ch/statement_conclusion}
В результате получена формальная постановка, включающая структуру решения (глобальный план), критерии оптимальности и набор ограничений выполнимости. В последующих главах данная постановка используется как основа для разработки алгоритма поиска решений и его программной реализации, а также для постановки и анализа вычислительных экспериментов.

\FloatBarrier
