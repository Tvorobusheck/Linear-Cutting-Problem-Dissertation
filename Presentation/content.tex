%=============================================================================
% ВВЕДЕНИЕ
%=============================================================================
\section{Введение}

\begin{frame}
    \frametitle{Актуальность темы исследования}
    \begin{itemize}
        \item Задача раскроя рулонных материалов~--- ключевая в~операционном управлении целлюлозно-бумажными предприятиями
        \item Неэффективный раскрой приводит к:
              \begin{itemize}
                  \item росту отходов материала (5--15\% стоимости сырья)
                  \item увеличению времени переналадок оборудования
                  \item разбалансировке загрузки производственных линий
              \end{itemize}
        \item Существующие модели не~учитывают одновременно:
              \begin{itemize}
                  \item многоуровневую структуру производства
                  \item многокритериальность задачи
                  \item динамику складских запасов
              \end{itemize}
    \end{itemize}
\end{frame}

\begin{frame}
    \frametitle{Цель и задачи исследования}
    \textbf{Цель:} разработка математической модели и~алгоритма решения задачи линейного раскроя рулонных материалов с~учётом многоуровневой структуры производства.
    
    \vspace{0.5cm}
    \textbf{Задачи:}
    \begin{enumerate}
        \item Формализовать производственную задачу раскроя с~учётом технологических ограничений
        \item Разработать многокритериальную математическую модель
        \item Разработать алгоритм решения на~основе генетического поиска
        \item Реализовать программный комплекс
        \item Провести экспериментальное исследование
    \end{enumerate}
\end{frame}

%=============================================================================
% ГЛАВА 1: ТЕОРЕТИЧЕСКАЯ ЧАСТЬ
%=============================================================================
\section{Постановка задачи}

\begin{frame}
    \frametitle{Технологический контекст}
    \textbf{Этапы производства:}
    \begin{enumerate}
        \item Подготовка целлюлозной массы
        \item Формование бумажного полотна на~БДМ
        \item Намотка полотна в~тамбуры (jumbo-рулоны)
        \item \textbf{Продольная резка тамбуров на~ПРС}
        \item Упаковка готовой продукции
    \end{enumerate}
    
    \vspace{0.3cm}
    \textbf{Схема производственного потока:}
    \[
    \text{БДМ} \xrightarrow{\text{тамбуры}} \text{Склад} \xrightarrow{\text{тамбуры}} \text{ПРС} \xrightarrow{\text{рулоны}} \text{Заказчик}
    \]
\end{frame}

\begin{frame}
    \frametitle{Основные понятия}
    \begin{itemize}
        \item \textbf{Заказ}~--- запрос на~рулоны определённой марки, ширины и~количества
        \item \textbf{Тамбур}~--- полуфабрикат фиксированной ширины (jumbo-рулон)
        \item \textbf{Раскрой}~--- схема продольной резки тамбура на~рулоны
        \item \textbf{Переналадка}~--- перестановка ножей на~ПРС при~смене раскроя
        \item \textbf{Глобальный план}~--- совокупность:
              \begin{itemize}
                  \item планов производства тамбуров на~БДМ
                  \item планов раскроя на~ПРС
                  \item календарного расписания операций
              \end{itemize}
    \end{itemize}
\end{frame}

\begin{frame}
    \frametitle{Критерии качества плана}
    \textbf{Векторная целевая функция:}
    \[
    \mu(G) = \langle \omega(G),\,\varphi(G),\,\tau(\overline{\overline{Y}}) \rangle \rightarrow \min
    \]
    
    \vspace{0.3cm}
    \textbf{Критерии (в порядке приоритета):}
    \begin{enumerate}
        \item $\omega(G)$~--- суммарная ширина отходов
        \item $\varphi(G)$~--- максимальное число раскроев на~одном ПРС
        \item $\tau(\overline{\overline{Y}})$~--- общее количество перестановок ножей
    \end{enumerate}
    
    \vspace{0.3cm}
    \textbf{Подход к~многокритериальности:} лексикографическая минимизация
\end{frame}

\begin{frame}
    \frametitle{Формальная постановка задачи}
    \textbf{Обозначения:}
    \begin{itemize}
        \item $M = \{1, \ldots, m\}$~--- множество БДМ
        \item $S = \{1, \ldots, s\}$~--- множество ПРС
        \item $O = \{1, \ldots, n\}$~--- множество заказов
        \item $W_j$~--- ширина тамбуров марки $j$
        \item $N_i$~--- максимальное число ножей на~ПРС $i$
    \end{itemize}
    
    \vspace{0.3cm}
    \textbf{Глобальный план:}
    \[
    G = (\overline{\overline{X}}, \overline{\overline{Y}}, \overline{\overline{Z}})
    \]
    \begin{itemize}
        \item $\overline{\overline{X}}$~--- планы производства тамбуров
        \item $\overline{\overline{Y}}$~--- планы раскроя
        \item $\overline{\overline{Z}}$~--- очередь событий (расписание)
    \end{itemize}
\end{frame}

\begin{frame}
    \frametitle{Ограничения задачи}
    \begin{enumerate}
        \item \textbf{Порядок марок:} на~БДМ марки производятся в~технологическом порядке
        \item \textbf{Допустимость раскроев:}
              \begin{itemize}
                  \item сумма ширин $\le$ ширина тамбура
                  \item число рулонов $\le$ число ножей
              \end{itemize}
        \item \textbf{Выполнение заказов:} все~рулоны должны быть нарезаны
        \item \textbf{Соблюдение сроков:} дедлайны заказов
        \item \textbf{Наличие тамбуров:} раскрой после производства
        \item \textbf{Вместимость складов:} ограничения буферов
    \end{enumerate}
\end{frame}

%=============================================================================
% ГЛАВА 2: АЛГОРИТМ РЕШЕНИЯ
%=============================================================================
\section{Алгоритм решения}

\begin{frame}
    \frametitle{Выбор метода решения}
    \textbf{Задача является NP-трудной:}
    \begin{itemize}
        \item Число раскроев: $O(n^{k_{\max}})$, где $k_{\max} = \lfloor W / w_{\min} \rfloor$
        \item При $n=50$, $k_{\max}=6$: $\approx 1.5 \cdot 10^{10}$ вариантов
    \end{itemize}
    
    \vspace{0.3cm}
    \textbf{Выбор: генетический алгоритм}
    \begin{itemize}
        \item Хорошо работает для~комбинаторных задач
        \item Естественное кодирование решений
        \item Гибкая настройка качество/время
        \item Учёт сложной структуры ограничений
    \end{itemize}
\end{frame}

\begin{frame}
    \frametitle{Представление решений}
    \textbf{Особь популяции} = глобальный план $G$
    
    \vspace{0.3cm}
    \textbf{Компоненты взаимосвязаны:}
    \begin{itemize}
        \item Изменение плана раскроя $\overline{\overline{Y}}$ требует:
              \begin{itemize}
                  \item пересчёта плана производства $\overline{\overline{X}}$
                  \item перегенерации расписания $\overline{\overline{Z}}$
              \end{itemize}
    \end{itemize}
    
    \vspace{0.3cm}
    \textbf{План раскроя $Y_i$:} упорядоченный список раскроев
    
    \textbf{Раскрой:} мультимножество ширин рулонов
\end{frame}

\begin{frame}
    \frametitle{Генетические операторы}
    \textbf{1. Мутация (Mutate):}
    \begin{itemize}
        \item Внутренняя перестановка рулонов в~раскроях
        \item Внешняя перестановка раскроев
        \item Сохраняет допустимость решения
    \end{itemize}
    
    \vspace{0.3cm}
    \textbf{2. Глобальная мутация (GlobalMutate):}
    \begin{itemize}
        \item Применение Mutate ко~всем планам раскроя
        \item Пересборка $\overline{\overline{X}}$ и $\overline{\overline{Z}}$
        \item Принятие только при~улучшении $\mu(G)$
    \end{itemize}
    
    \vspace{0.3cm}
    \textbf{3. Кроссинговер:} парный отбор лучшей особи
\end{frame}

\begin{frame}
    \frametitle{Алгоритм GeneticSearch}
    \begin{enumerate}
        \item \textbf{Инициализация:} генерация популяции $S_0$ размера $P$
        \item \textbf{Итерации:} пока $|S| > 1$:
              \begin{itemize}
                  \item разбить популяцию на~пары
                  \item к~каждой паре применить кроссинговер
                  \item сформировать новую популяцию из~победителей
              \end{itemize}
        \item \textbf{Результат:} лучшее найденное решение
    \end{enumerate}
    
    \vspace{0.3cm}
    \textbf{Сложность:} $O(P \log P \cdot Q \cdot q_{\max})$
    
    \textbf{Число поколений:} $\lceil \log_2 P \rceil$
\end{frame}

\begin{frame}
    \frametitle{Алгоритм генерации расписания}
    \textbf{GenerateEventQueue} моделирует динамику производства:
    
    \begin{enumerate}
        \item Инициализация: машины свободны, склады пусты, $t=0$
        \item Пока не~все операции выполнены:
              \begin{itemize}
                  \item определить доступные операции
                  \item выбрать по~приоритету (раскрой важнее производства)
                  \item запланировать, обновить состояние
              \end{itemize}
        \item Вернуть очередь событий
    \end{enumerate}
    
    \vspace{0.3cm}
    \textbf{Гарантирует:} соблюдение ограничений вместимости складов
\end{frame}

%=============================================================================
% ГЛАВА 3: ПРОГРАММНАЯ РЕАЛИЗАЦИЯ
%=============================================================================
\section{Программная реализация}

\begin{frame}
    \frametitle{Архитектура программного комплекса}
    \textbf{Технологический стек:}
    \begin{itemize}
        \item Python 3.10+
        \item Pandas~--- обработка данных
        \item OpenPyXL~--- работа с~Excel
        \item Tkinter~--- графический интерфейс
        \item pytest~--- тестирование
    \end{itemize}
    
    \vspace{0.3cm}
    \textbf{Модули:}
    \begin{itemize}
        \item CLI~--- интерфейс командной строки
        \item GUI~--- графический интерфейс
        \item Алгоритмический модуль
        \item Модельный слой
        \item Модуль ввода/вывода (CSV, Excel)
    \end{itemize}
\end{frame}

\begin{frame}
    \frametitle{Реализованные методы}
    \textbf{1. Полный перебор (Brute Force):}
    \begin{itemize}
        \item Гарантирует оптимальное решение
        \item Применим для $Q \le 20$--$30$ заказов
        \item Используется как эталон
    \end{itemize}
    
    \vspace{0.5cm}
    \textbf{2. Генетический алгоритм:}
    \begin{itemize}
        \item Основной метод для реальных задач
        \item Параметры по умолчанию: $P = 100$
        \item Решает задачи с сотнями заказов
    \end{itemize}
\end{frame}

%=============================================================================
% РЕЗУЛЬТАТЫ
%=============================================================================
\section{Результаты}

\begin{frame}
    \frametitle{Сравнение методов}
    \centering
    \begin{tabular}{lccc}
        \toprule
        \textbf{Заказов} & \textbf{Полный перебор} & \textbf{ГА} & \textbf{Отклонение} \\
        \midrule
        20  & 12 сек     & 0,1 сек  & 0\%  \\
        30  & 8 мин      & 0,3 сек  & 2\%  \\
        40  & 14 часов   & 0,8 сек  & 3\%  \\
        60  & ---        & 2 сек    & ---  \\
        100 & ---        & 8 сек    & ---  \\
        300 & ---        & 3 мин    & ---  \\
        500 & ---        & 12 мин   & ---  \\
        \bottomrule
    \end{tabular}
\end{frame}

\begin{frame}
    \frametitle{Влияние размера популяции}
    \centering
    \begin{tabular}{lcc}
        \toprule
        \textbf{Размер $P$} & \textbf{Качество} & \textbf{Время} \\
        \midrule
        10   & нестабильное & быстро \\
        50   & хорошее      & умеренно \\
        \textbf{100}  & \textbf{оптимальный баланс} & \textbf{оптимально} \\
        200  & незначительно лучше & медленно \\
        500  & незначительно лучше & очень медленно \\
        \bottomrule
    \end{tabular}
    
    \vspace{0.5cm}
    \textbf{Рекомендация:} $P = 100$ для большинства задач
\end{frame}

\begin{frame}
    \frametitle{Пример работы алгоритма}
    \textbf{Исходные данные:}
    \begin{itemize}
        \item 5 заказов: ширины 200--400~мм, всего 33 рулона
        \item 2 БДМ (ширина 1200 и 1500~мм)
        \item 3 ПРС (до 5--6 ножей)
        \item 2 склада (вместимость 8--10)
    \end{itemize}
    
    \vspace{0.3cm}
    \textbf{Результат ($P=10$, время 0,2 сек):}
    \[
    \mu(G) = \langle 104,\, 5,\, 10 \rangle
    \]
    \begin{itemize}
        \item Отход: 104~мм ($<$1\% ширины)
        \item Макс. раскроев на ПРС: 5
        \item Перестановок ножей: 10
    \end{itemize}
\end{frame}

\begin{frame}
    \frametitle{Практические результаты внедрения}
    \textbf{Программный комплекс внедрён в~опытную эксплуатацию:}
    
    \vspace{0.5cm}
    \begin{itemize}
        \item Сокращение времени планирования: \textbf{часы $\rightarrow$ минуты}
        \item Снижение отходов материала: \textbf{5--8\%}
        \item Повышение равномерности загрузки оборудования
        \item Уменьшение числа переналадок: \textbf{10--15\%}
    \end{itemize}
\end{frame}

%=============================================================================
% ЗАКЛЮЧЕНИЕ
%=============================================================================
\section{Заключение}

\begin{frame}
    \frametitle{Основные результаты работы}
    \begin{enumerate}
        \item Разработана \textbf{математическая модель} задачи линейного раскроя с~учётом многоуровневой структуры производства
        \item Предложена \textbf{многокритериальная постановка} с~лексикографической минимизацией
        \item Разработан \textbf{генетический алгоритм} с~операторами мутации, глобальной мутации и~парного отбора
        \item Создан \textbf{программный комплекс} с~CLI и GUI интерфейсами
        \item Проведено \textbf{экспериментальное исследование}, подтвердившее эффективность подхода
    \end{enumerate}
\end{frame}

\begin{frame}
    \frametitle{Научная новизна}
    \begin{enumerate}
        \item Комплексная модель, учитывающая:
              \begin{itemize}
                  \item многоуровневую структуру (БДМ~$\rightarrow$ склад~$\rightarrow$ ПРС)
                  \item многокритериальность (отходы, загрузка, переналадки)
                  \item динамику складских запасов
              \end{itemize}
        \item Модификация генетического алгоритма с~кроссинговером на~основе парного отбора
        \item Алгоритм построения расписания с~учётом ограничений вместимости
    \end{enumerate}
\end{frame}

\begin{frame}
    \frametitle{Публикации по теме диссертации}
    \textbf{Статьи в рецензируемых журналах (ВАК):}
    \begin{itemize}
        \item Клименко~В.А. Математическая модель задачи линейного раскроя... // Известия высших учебных заведений. Приборостроение, 2024
        \item Клименко~В.А. Генетический алгоритм решения задачи раскроя... // Инженерный вестник Дона, 2025
    \end{itemize}
    
    \vspace{0.3cm}
    \textbf{Конференции:}
    \begin{itemize}
        \item FRUCT-2025 (Scopus)
        \item Научная сессия ГУАП, Москва, 2024
        \item ИТНОП-2025, ПетрГУ
    \end{itemize}
    
    \vspace{0.3cm}
    \textbf{Свидетельство о регистрации программы:}
    \begin{itemize}
        \item Программа оптимизации раскроя рулонных материалов, 2024
    \end{itemize}
\end{frame}

\begin{frame}
    \frametitle{}
    \centering
    \Huge
    Спасибо за внимание!
    
    \vspace{1cm}
    \large
    Вопросы?
\end{frame}

